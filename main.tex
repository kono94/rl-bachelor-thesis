\documentclass[12pt]{scrartcl}
\usepackage{comment} % enables the use of multi-line comments (\ifx \fi) 
\usepackage[english,german]{babel} 
\usepackage[utf8]{inputenc}
\usepackage{fancyhdr}
\usepackage{graphicx}
\usepackage[article, total={6in, 8in}]{geometry}
\usepackage{float}
\usepackage{wrapfig}
\usepackage[onehalfspacing]{setspace}
\usepackage{url}
\usepackage{tabularx}
\usepackage{pdfpages}
\usepackage{wrapfig}
\usepackage[T1]{fontenc}% wichtig für Trennung von Wörtern mit UmlaPuten
\usepackage{microtype}% verbesserter Randausgleich
\setlength{\headheight}{21.4pt}
\usepackage{apacite}
\usepackage{natbib}
\usepackage{eurosym}
\usepackage{amsmath}
\usepackage{algorithm}
\usepackage{algorithmicx}
\usepackage[noend]{algpseudocode}

\begin{spacing}{1.0}
	\bibliographystyle{apacite}
\end{spacing}
\setlength{\parindent}{0em}
\setlength{\parskip}{2mm}

\makeatletter
\def\BState{\State\hskip-\ALG@thistlm}
\makeatother

\DeclareMathOperator{\EX}{\mathbb{E}}% expected value
\numberwithin{equation}{section}
% Special characters
\usepackage{eucal}
% math alphabet
\usepackage{amssymb}
\usepackage{scrpage2}%
\DeclareMathOperator*{\argmax}{argmax}
\DeclareMathOperator*{\max}{argmax}

%Pseudo-code indent
\algdef{SE}[SUBALG]{Indent}{EndIndent}{}{\algorithmicend\ }%
\algtext*{Indent}
\algtext*{EndIndent}

%%%% footer and header %%%%
\usepackage{scrpage2}%
\pagestyle{scrheadings}%  S
\clearscrheadfoot% 
\setheadwidth{text}%
\automark{section}% 
\ihead{\textbf{\pagemark}}
\renewcommand{\sectionmark}[1]{\markright{\ #1}} 
\ohead{\rightmark}
\setheadsepline{0.5pt}
%%%% \footer and header %%%%

\begin{document}

\begin{titlepage}
	\centering	
	{\scshape\LARGE Hochschule Bremerhaven \par}
	\vspace{1cm}
	{\scshape\Large Exposé für eine Bachelorarbeit zum Thema:\par}
	\vspace{1.5cm}
	{\huge\bfseries Reinforcement Learning\par}
	\vspace{2cm}
	{\Large\itshape Theoretische Grundlagen der tabellarischen Lernmethoden und praktische Umsetzung am Beispiel eines Ameisen-Agentenspiels
	\par}
	\vfill
	\begin{tabularx}{\textwidth}{lX}
		Autor: & Jan Löwenstrom \\
		Matrikelnr.: & 34937 \\
		Erstprüfer: & Prof. Dr.-Ing. Henrik Lipskoch \\
		Zweitprüfer: & Prof. Dr. Mathias Lindemann \\
	\end{tabularx}  
    \vfill

% Bottom of the page with current date
	{\large \today \par}       
\end{titlepage}

% Count title page as first page
\setcounter{page}{2}
\tableofcontents
\pagebreak
\listoffigures
\newpage


\noindent
\section{Einleitung}

Das ist eine Einleitung8us
\pagebreak

\section{Grundlagen}
	Dieser Teil der Arbeit gibt einen Überblick über sämtliche Bestandteile des Reinforcement Learnings. Dabei wird zunächst der wichtige mathematische Rahmen erläutert, der als Markov-Entscheidungsprozess verstanden wird. Aus diesem Rahmen lässt sich ein generisches Agent-Umwelt-Interface ableiten, auf welches eingegangen wird, um fundamentale Bestandteile, wie Belohnungen, Episoden, Gewinne und Nutzenfunktionen zu erläutern. 
\par
Abgerundet wird der Grundlagenteil mit der Auseinandersetzung von Strategien und dem Streben nach Optimalität, sowie der Darstellung des sog. \textit{Exploration-Exploitation-Dilemmas}.
	\subsection{Markov Entscheidungsprozess}
	\par 
Die Umwelt wird in den allermeisten Fällen als Markow-Entscheidungsprozess (\textit{Markov Decision Process, MDP}) definiert. //TODO Als \textit{MDP} versteht sich die Formalisierung von sequentiellen Entscheidungsproblemen, bei denen eine Entscheidung nicht nur die sofortige Belohnung beeinflusst, sondern auch alle Folgezustände und somit auch alle zukünftigen Belohnungen (S. 47). Zudem bieten sie den mathematischen Rahmen für das \textit{Reinforcement Learning} Problem, um z.B. Beweise über das Konvergenzverhalten eines Algorithmus hin zu einer optimalen Strategie führen oder andere theoretische Aussagen treffen zu können. Außerdem müssen Probleme die als \textit{MDP} definiert werden zugleich die Markow-Eigenschaft erfüllen, die von essentieller Bedeutung ist und in Kapitel X näher erläutert wird.
\par 

\begin{figure}[H]
    \centering
    \includegraphics[height=150px]{images/agentUmweltInterface.png}
    \caption{Agent-Umwelt Interface}
\end{figure}


Der Agent interagiert mit dem \textit{MDP} jeweils zu diskreten Zeitpunkten $t = 0, 1, 2, 3, \dots$. \\
Zu jedem Zeitpunkt $t$ beobachtet der Agent den Zustand seiner Umgebung $S_t \in \mathcal{S}$ und wählt aufgrund dessen eine Aktionen $A_t \in \mathcal{A}$. Als Konsequenz seiner Aktion erhält er einen Zeitpunkt später eine Belohnung $R_{t+1} \in \mathcal{R} \subset\mathbb{R} $ und stellt den Folgezustand $S_{t+1}$ fest. Das Zusammenspiel zwischen Agenten und MDP erzeugt also folgende Reihenfolge:
\[S_0, A_0, R_1, S_1, A_1, R_2, S_2, A_2, R_3, \dots\]

Wird einfach nur von \textit{MDPs} gesprochen, ist die endliche Variante (\textit{finite MDP}) gemeint, bei dem die Mengen der Zustände, Aktionen und Belohnungen ($\mathcal{S}, \mathcal{A}, \mathcal{R}$) eine endliche Anzahl an Elementen besitzen. In diesem Fall haben die zufälligen Variablen $R_t$ und $S_t$ wohl definierte diskrete Wahrscheinlichkeitsverteilungen, die nur von dem vorigen Zustand und vorigen Aktion abhängig sind (S.48). Die Wahrscheinlichkeit, dass die bestimmten Werte für diese Variablen $s' \in \mathcal{S}$ und $r \in \mathcal{R}$ eintreten, für einen bestimmten Zeitpunkt $t$ und dem vorigen Zustand $s$ und Aktion $a$, kann somit durch folgende Funktion beschrieben werden:

\[p(s',r \mid s,a) \doteq Pr\{S_t=s',R_t=r|S_{t-1}=s,A_{t_1}=a\},\]

für alle $s', s \in \mathcal{S}, r \in \mathcal{R}$ und $a \in \mathcal{A}(s)$. Diese Funktion p definiert die sog. Dynamiken (\textit{Dynamics}) eines \textit{MDP}. Sie ist eine gewöhnliche deterministische Funktion mit vier Parametern $p: \mathcal{S} \times \mathcal{R} \times \mathcal{A} \rightarrow [0,1]$. Das \glqq$\mid$\grqq{} Zeichen kommt ursprünglich aus der Notation für bedingte Wahrscheinlichkeiten, soll hier aber nur andeuten, dass es sich um eine Wahrscheinlichkeitsverteilung handelt für jeweils alle Kombinationen von $s$ und $a$:

\[ \sum_{s' \in \mathcal{S}} \sum_{r \in \mathcal{R}} p(s', r \mid s,a) = 1 \ \forall s \in \mathcal{S}, a \in \mathcal{A}(s)\]

Ist die Zustandsüberführungsfunktion nicht stochastisch, so ist $p$ immer nur für ein bestimmtes Triplet $(s,a,r)$ für jedes $s' \in \mathcal{S}$ gleich 1, für alle andere jeweils 0. Mit anderen Worten, wird im Zustand $s$ die Aktion $a$ gewählt, führt dies immer zu einem bestimmten Folgezustand $s’$. 
\par 

Das MDP Framework gilt als extrem flexibel und kann auf die unterschiedlichsten Probleme angewendet werden. Es bietet die nötige Abstraktion für Probleme, bei denen unter Vorgabe eines Ziels mittels Interaktionen gelernt wird. Dabei sind die Einzelheiten über eigentliche Ziel, die Zustände oder die Form des Agenten unerheblich, denn jedes zielgerichtete Lernen kann auf drei Signale reduziert werden, die zwischen dem Agenten und der Umwelt ausgetauscht werden. Ein Signal repräsentiert die Entscheidung, die der Agent getroffen hat (die Aktion), ein Signal repräsentiert die Basis, auf der er diese Entscheidung getroffen hat (der Zustand) und ein Signal definiert das zu erreichende Ziel (die Belohnung).



	\subsection{Markov-Eigenschaft und Zustandsmodellierung}
	Die Markov-Eigenschaft erhält ein eigenes Kapitel, da sie wichtig zum Verständnis dieser Arbeit ist und bei der Modellierung eines Reinforcement Learning Problems eine besondere Rolle spielt. Verbinden lässt sich dies sehr gut mit einem Einblick über die grundsätzliche Modellierung von Zuständen bei einem Reinforcement Learning Problem.

\begin{quote}
    The future is independent of the past given the present
  \end{quote}

Dieser Satz erscheint oft in der Literatur, wenn es um die Markov-Eigenschaft geht, so z.B. in den Arbeiten von \cite{Feldman2010}, \cite{kumar2014markov}, \cite{capela2019monogamy} und \cite{SaulMarkov}, oder auch in der Vorlesung der Stanford-Professorin Emma \cite{Brunskill}. Er fasst prägnant zusammen, was die Markov-Eigenschaft aussagt. Im Zusammenhang von MDPs lässt sich dieser Satz so übersetzen, dass ein Folgezustand nicht abhängig von Aktionen bzw. Zuständen in der Vergangenheit ist, sondern ausschließlich von dem aktuellen Zustand und der aktuell gewählten Aktion.
\par 
\cite{Sutton1998} sehen die Markov-Eigenschaft als Einschränkung für die Zustände und nicht für den Entscheidungsprozess als solches. 
Ausschlaggebend ist, dass der Zustand, auf dessen Basis der Agent seine Entscheidung trifft, alle notwendigen Informationen der Vergangenheit beinhaltet, die für die Zukunft relevant sind (S.49).
Die Umwelt ist somit nicht notwendigerweise gezwungen, Markov-konforme Zustände zu liefern. \cite{Brunskill} wählt aufgrund dessen die Bezeichnung \glqq Beobachtung\grqq{} (Observation $O_t$) als Feedback der Umwelt nach einer Aktion. Jene Beobachtungen können anschließend durch eine interne Repräsentation zu Markov-Zuständen verarbeitet werden, die dann dem Entscheidungsfinder zugrunde liegen.
\par
Folgendes Beispiel, basierend auf der Vorlesung von \cite{Brunskill}, liefert einen guten Einblick in die Zustandsmodellierung und der Problematik, die mit der Markov-Eigenschaft einhergeht.
\par 
\begin{figure}[H]
  \centering
  \includegraphics[height=200px]{images/2passagesDefault.png}
  \caption{ Zwei-Wege Beispiel zu der Markov-Eigenschaft}
  \label{fig:2-Wege-1}
\end{figure}

Gegeben ist ein beweglicher Roboter und eine Strecke mit zwei Korridoren. Der Roboter ist mit vier Sensoren ausgestattet, die jeweils eine Himmelsrichtung abdecken. Diese Sensoren sind in der Lage, angrenzende Wände zu erkennen und bilden den Zustand der Umwelt ab. Wahlweise ist der Zustand im Uhrzeiger definiert $\{N, O, S, W\}$, wobei 1 angibt, dass eine Wand erkannt wurde und 0, dass sich keine Wand in der unmittelbaren Nähe befindet. Es ergeben sich folglich 16 unterschiedliche Zustände, die der Agent unterscheiden und auf dessen Basis er Entscheidungen treffen kann (vier Aktionen: Fahrt in jeweils eine Richtung). Der Roboter soll sein Ziel erreichen, markiert mit einer Flagge, ohne dabei in eine der beiden Fallen zu navigieren.
\par
\begin{wrapfigure}{H}{0.5\textwidth}
  \begin{center}
  \includegraphics[height=200px]{images/2passagesStart.png}  \end{center}
  \caption{Zwei-Wege Beispiel Forts.}
  \label{fig:2-Wege-2}
\end{wrapfigure}

Eine potenzielle Startposition, wie in Abb. \ref{fig:2-Wege-1} dargestellt, liefert somit den Zustand $\{0, 1, 1, 1\}$. Angenommen der Agent hat gelernt in diesem Zustand Richtung Norden zu fahren, dann ist der Folgezustand ebenfalls $\{0, 1, 1, 1\}$. Schließlich erreicht er den ersten Korridor. Der westliche Sensor liefert folgerichtig 0 und der Zustand ist $\{0, 1, 0, 0\}$. Da der Agent nicht den ersten Korridor folgen darf, sondern dem zweiten, muss der Zustand $\{0, 1, 0, 0\}$ ebenfalls die Aktion \glqq nach Norden fahren\grqq{} auslösen. Das Besondere hier ist jedoch, dass der Zustand bei dem zweiten Korridor identisch mit dem Zustand bei dem ersten Korridor ist und der Agent somit keine Chance hat, zu unterscheiden, vor welchem er sich gerade befindet, siehe Abb. \ref{fig:2-Wege-2}. Er würde ebenfalls, wie schon bei dem ersten Korridor, weiter nach Norden und letztendlich in die Falle fahren.
\par 

Bezogen auf diesen Entscheidungsprozess ist die Modellierung der Zustände über den Sensorinput alleine nicht ausreichend, um die gestellte Aufgabe zu lösen. Die Kombination von Aufgabenstellung und dem Format der Zustände in dieser Form erfüllt insofern nicht die Markov-Eigenschaft, dass auf Basis der erkannten Zustände keine Möglichkeit besteht, die optimalen Entscheidungen zu treffen. 
\par

In der Theorie ist es jedoch möglich diesen Entscheidungsprozess als MDP umzumodellieren.
Dazu kann z.B. die gesamte Historie der Zustände und Aktionen gespeichert werden. Dadurch ist der Roboter in der Lage, zurückzuverfolgen, wo er sich zurzeit befindet. Obowhl dies rein theoretisch möglich ist, sollte diese Herangehensweise vermieden werden. Historien als Zustand für eine Entscheidung zu betrachten bedient zwar die Markov-Eigenschaft, ist allerdings in der Praxis nicht praktikabel, da der Zustandsraum auf diese Weise sehr schnell zu große Ausmaße annimmt.
\par 
Eine weitere Möglichkeit besteht darin, die Sensordaten als Beobachtungen der Umwelt zu betrachten und eine interne Repräsentation von Markov-Zuständen zu pflegen. Dabei bildet der Agent die Umwelt nach jeder erhaltenden Beobachtung suk­zes­si­ve nach, wodurch letztendlich ein Gesamtbild der Umgebung entsteht. Diese Form der Zustandsbildung bedarf jedoch zusätzlicher Algorithmen, die zwischen der Wahrnehmung des Agenten und dem jeweiligen RL-Algorithmus sitzen. 
\par 
Zudem ist das Lernen auf diese Weise auf eine bestimmte Umgebung festgelegt. Wird der Roboter mit einer neuer Welt konfrontiert, z.B. eine Welt mit drei Korridoren, so kann er Gelerntes nicht anwenden, weil seine interne Repräsentation invalide ist.
\par 
Letztendlich muss jedes Szenario zu Beginn genau untersucht werden, um zu beurteilen, ob Reinforcement Learning überhaupt auf dieses Problem anwendbar ist. Dabei spielt vor allem die Markov-Eigenschaft eine wichtige Rolle, die Grundvorraussetzung für alle RL-Algorithmen ist. Für eine Bewertung, ob optimales Verhalten durch gegebene Informationen erreicht werden kann,existieren jedoch keine festen Regeln oder eine Blaupause. Es muss auf Erfahrungswerte oder Untersuchungen zu dem Konvergenzverhalten und den Ergebnissen des gelernten Verhaltens zurückgegriffen werden, wie es auch in den Beispielen dieser Arbeit geschieht in dem Kapitel \ref{sec:praktischerTeil}.

	\pagebreak

	\subsection{Belohnungen und Zielstrebigkeit}
	Das Besondere an dem Reinforcement Learning ist das Belohnungssignal (\textit{Reward}), welches der Agent nach jeder Aktion erhält. Zu jedem diskreten Zeitpunkt wird dem Agenten eine Belohnung in Form einer einfachen Zahl $R_t \in \mathbb{R}$ zugestellt. Aufgabe eines jeden RL-Algorithmus ist es, die Summe aller gesammelten Belohnungen zu maximieren. Dabei ist entscheidend, dass der Fokus nicht ausschließlich auf die sofortigen Belohnungen gerichtet ist, sondern auf die erwartbare Summe aller Belohnungen über einen langen Zeitraum. Entscheidungen, die in der Gegenwart eine hohe sofortig Belohnung versprechen sind verführerisch, können sich aber in der Zukunft in Bezug auf den gesamten Prozess als suboptimal herausstellen. \cite[~S.53]{Sutton1998}
\par 
Eine Belohnungsfunktion wird in der Regel von einem Menschen definiert und hat den größten Einfluss darauf, wie der Agent sich verhalten soll. Die Festlegung von Belohnung bei bestimmten Events ist die einzige Möglichkeit, die der Agent hat, zu verstehen, welches Ziel er verfolgen soll. Somit ist die Modellierung der passenden Belohnungsfunktion zur korrekten Abbildung der eigentlichen Aufgabenstellung von gravierender Bedeutung.
\par 
Grundsätzlich gibt es zwei Ansätze, um eine Belohnungsfunktion zu formulieren. Verständlich werden diese durch ein Beispiel, bei dem ein Agent lernen soll, eine Partie Schach zu gewinnen. Die erste Möglichkeit besteht darin, dem Agenten ausschließlich eine Belohnung aufgrund des Spielausgangs zu geben. Er erhält +1, wenn er gewinnt, -1 bei einer Niederlage und 0 bei Unentschieden. Auf den ersten Blick erscheint dieser Ansatz trivial, ist aber die direkte Übersetzung des Ziels in eine Belohnungsfunktion. Die größte erwartbare Summe aller Belohnungen erzielt der Agent nur, wenn er lernt, das Spiel zu gewinnen. Größter Nachteil dieser Methode ist allerdings, dass der Agent keinerlei Hilfe oder Richtung bei dem Erkunden des Spiels erhält. Je größer der Zustands- und Aktionraum ist, desto länger braucht er um überhaupt einmal ein Spiel gewinnen zu können und zu lernen, welche Aktionen vorteilhaft sind und welche nicht.
\par 
Um dem entgegenzuwirken, werden dem Agenten bei der zweiten Möglichkeit feingranularere Belohnungen mitgeteilt, statt diese ausschließlich auf das Endresultat zu reduzieren. Bestimmte Belohnungen zeigen dann, ob der Agent seinem Ziel näher gekommen ist oder eine ungünstige Entscheidung getroffen hat. Zum Beispiel könnte dem Agenten eine hohe Belohnung von +10 gegeben werden, wenn er die gegnerische Dame aus dem Spiel nimmt. Es ist auch denkbar, dass jede Spielfeldkonstellation bewertet wird. Dieser Ansatz benötigt somit spezielles Vorwissen über das Problem und kann sich zugleich sehr negativ auf das Verfolgen des eigentlichen Ziels auswirken. Der Agent könnte zum Beispiel nur lernen in jedem Spiel die Dame des Gegners zu schlagen und dabei trotzdem immer die Partie zu verlieren.
\par 
Die korrekte Modellierung der Belohnungsfunktion hat somit eine besondere Bedeutung. \cite{Sutton1998} sind der Meinung, dass ein Schachagent nur angesichts des Spielaussgangs bewertet werden sollte und nicht aufgrund von Zwischenzielen wie z.B. dem Herausnehmen einer gegnerischen Spielfigur oder der Kontrolle über das Zentrum des Spielfelds (S.~53). 
\par 
Für eine korrekte Übersetzung der Aufgabenstellung zu einer geeigneten Belohnungsfunktion gibt es keine klaren, formalen Regeln. Ein*e Designer*in muss auf Erfahrungswerte und einen gewissen Grad an Kreativität zurückgreifen. Soll ein Agent z.B. ein Labyrinth durchlaufen und so schnell wie möglich hinausfinden, dann muss nach jeder Aktion eine negative Belohnung von -1 verteilt werden. Somit wird der Agent gezwungen, auf direktem Wege den Ausgang zu erreichen. Würde lediglich für das Erreichen des Ausgangs eine positive Belohnung vergeben werden, dann wäre die Summe aller Belohnungen für jede Abfolge von Aktionen gleich. Der Agent \glqq trödelt\grqq{}. Hat er durch Zufall aus dem Labyrinth gefunden, so könnte er bei weiteren Durchläufen keinen effektiveren Weg finden, denn für ihn haben alle Aktionsfolgen den gleichen Nutzen.
\par
Prinzipiell gilt, dass \glqq das Belohnungssignal dazu dient, dem Agenten mitzuteilen \textit{was} er erreichen soll, nicht \textit{wie} er es erreichen soll\grqq{} \cite[S.~54]{Sutton1998}.

	\subsection{Gewinn und Episoden}\label{Gewinne}
	In Kapitel 2.1 wurde gezeigt, dass die Interaktion eines Agenten mit seiner Umwelt als bestimmte Abfolge beschrieben werden kann \eqref{eq:episode}. In ihr werden letztendlich alle Triple von Zustand, ausgeführter Aktion aufgrund dieses Zustands und anschließende Belohnung chronologisch aufgezeichnet. Ist diese Reihenfolge endlich, so wird sie auch als Episode (\textit{Episode}) bezeichnet. Eine Episode fasst somit alle Informationen zusammen, die ein Agent erlebt, während er von einem beliebigen Startzustand aus anfängt die Umwelt zu erkunden. Das Ende einer Episode wird durch das Erreichen eines beliebigen Zielzustands erreicht. Ist eine Episode zu Ende, dann wird das Szenario zurückgesetzt und der Agent startet erneut im Startzustand. Episoden sind komplett unabhängig voneinander und erzeugen Abfolgen, die nicht durch vorrige Episoden beeinflusst sind.
\par
Bisher wurde erwähnt, dass das Ziel eines Agenten sei, die Summe der zu erwartenden Belohnungen zu maximieren. Formal betrachtet, versucht er somit die Sequenz der Belohnungen, die er nach dem Zeitpunkt $t$ erhält, den sog. erwarteten Gewinn (\textit{Return}), zu maximieren. Im einfachsten Fall sieht $G_t$ wie folgt aus, wobei $T$ der finale Zeitstempel ist \cite[S.55]{Sutton1998}:

\begin{equation}\label{eq:simpleReturn}
    G_t = R_{t+1} + R_{t+2} + R_{t+3} + \dots + R_{T}
\end{equation}

Bei episodialen Problemen lässt sich der Gewinn durch diese Addition von nachfolgenden Belohnungen für jeden Zeitpunkt $t$ ermitteln. Grund hierfür ist, dass wärend der Berechnung, nach Abschluss der Episode, alle Belohnungen bekannt sind. 
\par 
Jedoch existieren auch Probleme, die keine Endzustände definiert haben und daher einen sog. unendlichen Zeithorizont (\textit{infinite horizon}) besitzen. Sie lassen sich nicht in natürliche Sequenzen unterteilen und werden auch mit \glqq kontinuierlich\grqq{} betitelt, wobei dadurch ausschließlich beschrieben wird, dass die Interaktion zwischen Agenten und Umwelt kein definiertes Ende besitzt, die Zeitstempel sind weiterhin diskret. Folglich ist $T=\infty$, was wiederum bedeutet, dass der Gewinn unendlich ist.
\par 
Um diese kontinuierlichen und die zuvor beschriebenen episodialen Aufgaben im Bezug auf den Gewinn zu vereinheitlichen, wird das Konzept der Diskontierung (\textit{discounting}) verwendet. Dabei gibt der Parameter $\gamma$, $0\leq \gamma \leq 1$, Auskunft darüber, wie die Gewichtung zwischen sofortigen und zukünftigen Belohnungen verteilt ist. Der zukünftige diskontierte Gewinn, der durch die Aktion $A_t$ maximiert werden soll, berechnet sich somit wie folgt \cite[S.55]{Sutton1998}:

\begin{equation}\label{eq:discountedReturn}
    G_t = R_{t+1} + \gamma R_{t+2} + \gamma^2 R_{t+3} + \dots  = \sum_{k=0}^\infty{\gamma^k R_{t+k+1}}
\end{equation}
Eine wichtige Erkenntnis ist, dass Gewinne aufeinanderfolgender Zeitpunkte in Verbindung stehen. Vor allem Algorithmen, die nach jedem Zeitstempel updaten, profitieren von dieser Eigenschaft. Sie verwenden den geschätzten Gewinn des Folgezustands, also $G_{t+1}$, zur Berechnung von $G_t$, dem geschätzten Gewinn des aktuellen Zustands. Dieses Verfahren, bei dem ein Schätzwert aufgrund eines anderen Schätzwertes aktualisiert wird, wird auch als \textit{bootstrapping} bezeichnet.  
\par 
Durch simple Umformung wird der Zusammenhang von Gewinnen deutlich \cite[S.55]{Sutton1998}:

\begin{equation}\label{eq:successiveReturn}
    \begin{aligned}
    G_t &= R_{t+1} + \gamma R_{t+2} + \gamma^2 R_{t+3} + \gamma^3 R_{t+4} + \dots \\
    &= R_{t+1} + \gamma (R_{t+2} + \gamma R_{t+3} + \gamma^2 R_{t+4} + \dots)  \\
   & = R_{t+1} + \gamma G_{t+1}
    \end{aligned}
\end{equation}
Ist $\gamma = 0$, dann wählt der Agent seine Aktionen ausschließlich aufgrund der sofortigen Belohnung $R_{t+1}$. Je näher $\gamma$ an 1 ist, desto \glqq weitsichtiger\grqq{} wird der Agent, da der Gewinn für den Zeitpunkt $t$ sich zusätzlich aus zukünftigen Belohnungen zusammensetzt. $\gamma = 1$ führt zu der gleichen Summe wie \eqref{eq:simpleReturn} und wird bei Problemen bestimmt, die Episoden erzeugen. Dadurch trifft der Agent seine Entscheidungen immer aufgrund jeglicher Konsequenzen in der Zukunft bzw. bis zum Ende der jeweiligen Episode. Um zu erreichen, dass die unendliche Summe in \eqref{eq:discountedReturn} bei kontinuierlichen Aufgaben einen endlichen Wert annimmt, muss $\gamma < 1$ gegeben sein.
\par 
Probleme mit unendlichem Zeithorizont können durch die Vergabe einer künstlichen Schranke zu einer episodialen Aufgabe umformuliert werden. Denkbar z.B. durch die Festlegung der maximalen Anzahl an Aktionen oder besuchten Zustände. 
\par 
//TODO TD-Episodic tasks?! Weglassen?
\par
Die Algorithmen der Monte-Carlo-Methoden, die in Kapitel \ref{sec:MC} vorgestellt werden, können ausschließlich auf Basis von Episoden lernen. Jedoch existieren auch Methoden, wie das Temporal-Difference-Learning, siehe Kapitel \ref{sec:TD}, die neben dem episodialen Lernen, zusätzlich in der Lage sind, mit kontinuierlichen Aufgaben zurechtzukommen. 

	\subsection{Strategie und Nutzenfunktion}
	Fast alle Lernalgorithmen des Reinforcement Learning versuchen eine sog. Nutzenfunktion (\textit{value function}) zu schätzen. Diese Funktion sagt aus, "wie gut" es ist, dass sich der Agent in einem bestimmten Zustand befindet oder eine bestimmte Aktion in einem Zustand auszuführen. Das "wie gut" bezieht sich darauf,
welche Belohnungen in der Zukunft erwartbar sind, also wie der erwartete Gewinn ist. Zukünftige Belohnungen sind natürlicherweise abhängig davon, wie sich der Agent verhalten bzw. welche Entscheidungen er treffen wird. Nutzenfunktion sind deshalb immer in Bezug auf eine bestimmte Strategie definiert\cite[S.~58]{Sutton1998}.
\par 
Eine Strategie (\textit{Policy}) ist die Abbildung von Zuständen auf die Wahrscheinlichkeiten einzelne Aktion auszuführen. Folgt der Agent einer Strategie $\pi$ zum Zeitpunkt $t$, dann gibt $\pi(a\mid s)$ an, mit welcher Wahrscheinlichkeit $A_t = a$ ausgeführt wird, wenn $S_t = s$ \cite[S.~58]{Sutton1998}. Neben solchen stochastischen Strategien, existieren auch simplere, deterministische Strategien, die jedem Zustand nur eine Aktion zuordnen $\pi (s) = a$. 
\par
Wie anfangs erwähnt, gibt es zwei Varianten der Nutzenfunktion. Die erste sagt aus, wie groß der erwartete Gewinn für den Zustands $s$ ist, wenn in diesem gestartet und anschließend aufgrund der Strategie $\pi$ gehandelt wird. Dieser \textit{Zustands-Nutzen} kann für alle $s \in \mathcal{S}$ definiert werden:
\begin{equation}\label{eq:valueFunction}
    v_\pi(s) = \EX_\pi[G_t \mid S_t = s] = \EX_\pi[\sum_{k=0}^\infty{\gamma^k R_{t+k+1} \mid S_t = s}]
\end{equation}

Die zweiten Variante gibt Auskunft darüber wie groß der Nutzen ist, wenn im Zustand $s$ gestartet, daraufhin die Aktion $a$ ausgeführt und anschließend der Stragie $\pi$ gefolgt wird. $q_\pi$ wird auch als Aktion-Nutzen-Funktion für die Stragie $\pi$ bezeichnet und wird formal wie folgt definiert:
\begin{equation}\label{eq:actionValueFunction}
    q_\pi(s,a) = \EX_\pi[G_t \mid S_t = s, A_t = a] = \EX_\pi[\sum_{k=0}^\infty{\gamma^k R_{t+k+1} \mid S_t = s, A_t = a}]
\end{equation}

//TODO der Erwartungswert bezieht sich auf was?
//TODO functionsapproximation hier? 

	\subsection{Optimalität} \label{sec:optimality}
	Wofür Nutzenfunktionen?
Beste Strategie, beste Nutzenfunktion.
Warum actionValue besser ist (pefektes Modell)
Bellmann equation, Berechnung.
Approximation  - Cliffhänger zu den Methoden
Prediction und Control.


Ein Reinforcment Learning Problem zu lösen bedeutet, eine Strategie zu finden, die den größten Gewinn bringt. Dabei lassen sich Strategien vergleichen, insofern, dass eine Strategie besser ist als eine andere, wenn der erwartete Gewinn für alle Zustände größer oder gleich ist. Mit anderen Worten, $\pi \geq \pi'$ gilt, wenn $v_\pi(s) \geq v_{\pi'}(s)$ für alle $s \in \mathcal{S}$. Es existiert immer eine Stragie die besser oder gleich gegenüber allen anderen Strategien ist. Diese ist die optimale Strategie $\pi_*$. Optimale Strategien teilen die selbe (optimale) Zustands-Nutzenfunktion $v_*$ und (optimale) Aktions-Zustands-Nutzenfunktion $q_*$. 

\begin{equation}\label{eq:optimaleValueFunction}
    v_*(s) = \max_\pi v_\pi(s)
\end{equation}
\begin{equation}\label{eq:optimaleActionValueFunction}
    q_*(s,a) = \max_\pi q_\pi(s,a)
\end{equation}


Optimale Nutzenfunktionen sind solche, die den Gewinn im Bezug auf die Dynamiken des \textit{MDP} perfekt wiederspiegeln. Ist ein Modell der Umgebung vorhanden, dann ist es möglich, die optimale Nutzenfunktion zu berechnen, denn sich kann als Gleichungssystem verstanden werden, welches eine eindeutige Lösung hat. Dieses Gleichungssystem wird auch als \textit{Bellman Optimality Equation }bezeichnet \eqref{eq:bellman}, wird jedoch im Weiteren nicht geanuer erläutert. Grund hierfür ist, dass zur Lösung ein perfektes Modell vorhanden sein muss, eine Vorraussetzung, die unter normalen Umständen nicht oft gegeben ist. Selbst wenn die Dynamiken bekannt sind, kann die benötigte Rechenzeit zur Lösungen jedoch utopische Ausmaße annehmen. Das Gleichungssystem besitzt eine Gleichung für jeden Zustand, das beudetet, wenn ein Problem $n$ Zustände hat, ergeben sich $n$ Gleichungen mit $n$ Unbekannten \cite[S.~64]{Sutton1998}. 
\par
Bei einem Spiel wie "Backgammon" sind die Regeln bekannt, ein perfektes Modell ist somit vorhanden, aber es existieren $10^23$ Zustände, was die mathematische Berechnung von $v_*$ mittels der \textit{Bellman Optimality Equation} praktisch unmöglich macht. Dennoch stellt sie ein wichtiges Fundament des Reinforcment Learning dar, da die meisten Reinforcment Learning Algorithmen als annährendes Lösungsverfahren verstanden werden können \cite[S.~66]{Sutton1998}.
\par 
Die optimale Strategie lässt sich leicht ermitteln, wenn eine optimale Nutzenfunktion gegeben ist. Ist zum Beispiel $v_*$ gegeben und befindet sich der Agent in Zustand $s$, dann muss er eine Aktion vorrausschauen, um den Folgezustand $s'$ zu finden, der den maximalen Nutzen hat. Dieses Vorrausschen benötigt jedoch ebenfalls ein perfektes Modell der Umgebung, um die Übergange für jede Aktion zu berechnen. Das ist der ausschlaggebende Grund, warum in der Regel $q_*$ berechnet wird. Denn dieser Nutzen umfasst implizit den Nutzen der Folgezustand für jede Aktion. Somit muss der Agent im Zustand $s$ nur schauen, welche Aktion $a$ und somit welches Zustands-Aktions-Paar den größten Nutzen hat und wählt genau jene Aktion.
\par 


	\subsection{Generalized Policy Iteration}\label{sec:GPI}
	Bei der Berechnung einer optimalen Strategie $\pi_*$ spielt ein zugrundeliegendes Konzept  bei jeglichen Lernmethoden eine wichtige Rolle, die sog. \textit{Generalized Policy Iteration (GPI)}. Dieses, durch \cite{Sutton1998} geprägte, Prinzip beschreibt die Interaktion von zwei nebenläufigen Prozessen (S.~86). Ein Prozess sorgt dafür, dass die Nutzenfunktion beständig für die aktuelle Strategie wird. Er versucht das sog. \textit{Prediction Problem} zu lösen, bei dem die Nutzenfunktion $v_{\pi}$ oder $q_{\pi}$ geschätzt werden muss \cite[S.~18]{Wiering}. Jener Prozess wird als \textit{Policy Evaluation} bezeichnet und unterscheidet sich je nach verwendeten Lernverfahren. \textit{Model-based} Lernmethoden, bei denen ein perfektes Modell vorhanden ist, können den Nutzen für eine Strategie entweder direkt oder iterativ berechnen \cite[S.~18]{Wiering}. Hingegen benötigt die große Gruppe der \textit{model-free} Methoden die gesammelte Erfahrung durch eine Interaktion mit der Umwelt. Hierbei konvergiert der geschätzte Gewinn zu dem tatsächlichen Gewinn, solange jedes Zustands-Aktions-Paar unendlich oft besucht wird. Die Konvergenz lässt sich durch das \glqq Gesetz der großen Zahlen\grqq{} (\textit{Law of large numbers}) begründen \cite[S.~94]{Sutton1998}.
Ebendies sagt aus, dass die relative Häufigkeit eines Zufallsergebnisses bei zunehmender Anzahl der Ausführungen gegen die theoretische Wahrscheinlichkeit konvergiert. Für einen kompletten mathematischen Beweis siehe \cite[S.~181-189]{dekking2006modern}.
\par 
Das Wissen über den Nutzen der aktuellen Strategie $\pi$ wird von dem zweiten Prozess genutzt, um eine verbesserte Strategie $\pi'$ zu finden. Folgerichtig wird dieser Prozess als \textit{Policy Improvement} betitelt, der das sog. \textit{Control Problem} zu lösen versucht \cite[S.~18]{Wiering}.
\par 
GPI an sich beschreibt lediglich, dass diese zwei Prozesse miteinander Interagieren. Dabei konkurrieren sie auf der einen Seite, weil sie in unterschiedliche Richtungen ziehen. Eine Verbesserung der Strategie bei dem \textit{Policy Improvement}, indem die Strategie gierig im Bezug auf die Nutzenfunktion gemacht wird, führt dazu, dass die evaluierte Nutzenfunktion für die verbesserte Strategie inkorrekt wird \cite[S.~86]{Sutton1998}. Die erneute Evaluierung der Nutzenfunktion bei der \textit{Policy Evaluation} sorgt indirekt dafür, dass die Strategie nicht mehr gierig ist \cite[S.~86]{Sutton1998}.
\par 
\begin{figure}[H]
    \centering
    \includegraphics[width=0.7\textwidth]{images/gpi.jpg}
    \caption{GPI nach \cite[S.~86f]{Sutton1998}}
    \label{fig:GPI}
\end{figure}

Auf der anderen Seite kooperieren sie jedoch in dem Sinne, dass beide Prozesse sich nur dann stabilisieren, wenn eine Strategie durch eine eigens evaluierte Nutzenfunktion gefunden worden ist, die zugleich gierig im Bezug auf genau diese ist, siehe Abb. \ref{fig:GPI}. Sie haben somit ein gemeinsames Ziel; die optimale Nutzenfunktion bzw. die optimale Strategie zu finden.

	\subsection{Exploration-Exploitation Dilemma}
	Durch die Vergabe von Belohnungen und dem übergeordneten Ziel eines Agenten so viele Belohnungen wie möglich zu sammeln, ergibt sich eine spezielle Problematik bei dem Reinforcement Learning, die bei anderen Lernmethoden des Maschinellen Lernens nicht vorhanden ist. Um den Gewinn zu maximieren, muss der Agent auf der einen Seite Aktionen bevorzugen, die sich in der Vergangenheit bereits als gut herausgestellt haben. Er nutzt unvollständige Erfahrung, um so ausbeuterisch wie möglich zu handeln (\textit{Exploitation}). Andererseits ist der Agent dazu gezwungen, neue Aktionen auszuprobieren, damit der Zustands- und Belohnungsraum weiter erkundet wird, um bessere oder sogar optimale Entscheidungen in der Zukunft treffen zu können (\textit{Exploration}). 
\par 
Das Dilemma besteht darin, dass weder Exploration noch Exploitation ausschließlich verfolgt werden kann, ohne dabei die eigentliche Lernaufgabe zum Scheitern zu bringen. Dieses Exploration-Exploitation-Dilemma wird von Mathematikern seit Jahrzehnten intensiv untersucht, bleibt allerdings ungelöst \cite[S.~3]{Sutton1998}. Grundsätzlich muss ein Entscheidungsfinder  eine Reihe von unterschiedlichen Aktionen ausführen und zunehmend jene bevorzugen, die sich als gut herausstellen. Dementsprechend muss eine Balance zwischen den beiden Prozessen gefunden werden.
Eine Strategie, die ausschließlich ausbeuterisch handelt, wird auch als gierig (\textit{greedy}) bezeichnet. Der Begriff \glqq gierig\grqq{} bezeichnet in der Informatik eine Vorgehensweise, bei der immer die, zum Zeitpunkt der Wahl, vermeintlich beste Entscheidung getroffen wird \cite[S.~203]{greedy}. Dabei wird die Suche nach einem globalen Maximum komplett vernachlässigt. Auf den Kontext des Reinforcement Learning übertragen, wählt eine gierige Strategie für jeden Zustand immer jene Aktion, die den derzeitigen größten Nutzen besitzt. Nur wenn die Nutzenfunktion zu einer optimalen Nutzenfunktion konvergiert ist, ist eine gierige Nutzenfunktion auch gleichzeitig die optimale Strategie. Um jedoch die optimale Nutzenfunktion zu finden, muss erkundet werden.
\par
Ein trivialer, aber dennoch effektiver Ansatz ist es, die meiste Zeit gierig zu handeln, aber mit einer geringen Wahrscheinlichkeit $\epsilon$ eine zufällige Aktion auszuführen. Dabei spielen die geschätzten Nutzen der Aktionen keine Rolle und jede Aktion hat die gleiche Wahrscheinlichkeit ausgewählt zu werden. Zu vermerken ist, dass die gierige Aktion $A_*$ ebenfalls in der Menge $\mathcal{A}(S_t)$ enthalten ist. Solche Strategien werden entsprechend als $\epsilon-greedy$ bezeichnet \cite[S.~28]{Sutton1998}:

\begin{equation}\label{eq:greedyProbs}
    \pi(a|S_t) =   
        \begin{cases}
            1-\epsilon + \epsilon / |\mathcal{A}(S_t)|      & \quad \text{wenn } a = A_* \\
            \epsilon / |\mathcal{A}(S_t)|  & \quad \text{wenn } a \neq A_*
        \end{cases}
\end{equation}

	\subsection{Zusammenfasung}
	Das Reinforcement Learning wird duch den theoretischen Rahmen der \textit{Markov-Entscheidungsprozesse} beschrieben. Dabei interagiert ein Softwareagent zu diskreten Zeitstempeln mit seiner Umgebung. Als Reaktion auf seine gewählte \textit{Aktion} $a$ zum Zeitpunkt $t$ erhält er einen Zeitpunkt später eine \textit{Belohnung} $r$ und den veränderten \textit{Zustand} $s$ der \textit{Umgebung}.
\par 
Wird die Interaktion durch einen Terminalzustand beendet, so handelt es sich um ein Problem, welches \textit{Episoden} erzeugt. Existiert kein Terminalzustand und somit ein unendlicher Zeithorizont, dann ist dies ein \textit{kontinuierliches} Problem.
\par 
Das Ziel eines jeden RL-Algorithmus ist es, die Summe aller erhaltenen Belohnungen, den sog. \textit{Gewinn}, auf lange Sicht zu maximieren. Für episodale Probleme wird der Gewinn durch die Summe aller erhaltenen Belohnungen pro Episode gebildet. Bei kontinuierlichen Problemen ist dies jedoch nicht möglich, da diese Summe unendlich ist. Aus diesem Grund wird der \textit{Diskontierungsfaktor} $\gamma$ eingeführt. Je kleiner $\gamma$ gewählt ist, desto \glqq kurzsichtiger\grqq{} wird der Agent, da Belohnungen der Zukunft nicht mehr Teil des Gewinns sind.
\par 
Um optimales Verhalten zu erreichen, spielen vor allem zwei Konstrukte eine Rolle. Zum einen der sog. \textit{Nutzen}, der aussagt, wie \glqq gut\grqq{} ein Zustand $q(a)$ oder ein Zustands-Aktions-Paar $q(s,a)$ ist. \glqq Nutzen\grqq{} ist eine Schätzung über die tatsächliche, erwartbare Summe aller Belohnungen, die der Agent erhält, nachdem er sich in einem bestimmten Zustand befindet oder eine bestimmte Aktion in einem Zustand ausgeführt hat. Bei den tabularen Lernmethoden werden alle Werte der Nutzen in einer \glqq Tabelle\grqq{} (oder \textit{Map}) gespeichert, wodurch der Name zustande kommt.  
Zum anderen die \textit{Strategie} $\pi$, die Zustände auf Aktionen mappt und die Entscheidungen bzw. das Verhalten des Agenten steuert. In dieser Arbeit wird hauptsächlich eine $\epsilon$\textit{-greedy} Strategie verwendet, die mit einer Wahrscheinlichkeit von $\epsilon$ eine zufällige Aktion auswählt und mit einer Wahrscheinlichkeit von $1-\epsilon / |\mathcal{A}|$ jene Aktion, die aktuell den höchsten Nutzen besitzt. Diese $\epsilon$\textit{-greedy} Strategie ist zudem ein Lösungsansatz zur Vermeidung des sog. \textit{Exploration-Exploitation-Dilemmas}, da sie sowohl den Zustands- und Aktionsraum erkundet, als auch gierig handelt, um den größten Gewinn zu erhalten.
\par 
Eines der wichtigsten Konzepte, wenn es um Markov-Entscheidungsprozesse und der Modellierung der Zustände für das Reinforcement Learning geht, ist die sog. Markov-Eigenschaft. Diese sagt aus, dass ein Folgezustand nicht abhängig von Aktionen und Zuständen in der Vergangenheit ist, sondern ausschließlich von dem aktuellen Zustand und der aktuell gewählten Aktion. Ein Zustand muss folgerichtig alle nötigen Informationen der Vergangenheit beinhalten, die notwendig sind, um eine optimale Entscheidung treffen zu müssen.

\section{Lernmethoden}
	\subsection{Dynamische Programmierung}
	Die Algorithmen der Dynamischen Programmierung (\textit{Dynamic Programming}, DP) sind im Rahmen dieser Arbeit nicht implementiert und weiter untersucht worden. Dennoch ist ein grundlegendes Verständnis für die DP von Vorteil, da elementare Bestandteile auch in den nachfolgenden Kapiteln zu den Monte-Carlo Methoden und dem Temporal-Difference Learning referenziert werden.
\par 
Die grundlegende Idee der Dynamischen Programmierung ist die Aufteilung eines Optimierungsproblems in Teilprobleme. Dabei wird mit einem trivialem Problem gestartet und die optimale Lösung für jenes in einer Tabelle gespeichert, welches anschließend für die Lösung eines sukzessiv größer werdendes Problem verwendet wird \cite[S.~243]{mehlhorn}.
\par 
Bezogen auf das Reinforcement Learning, ist dieses Problem die Suche nach der optimalen Strategie $\pi_*$ bzw. der Lösung der \textit{Bellman Optimality Equation}, siehe \ref{eq:bellmanValue}. Wie bereits in Kapitel \ref{sec:optimality} erwähnt, ist es möglich dieses Gleichungssystem zu lösen und somit $v_*$ oder $q_*$ linear zu berechnen. Mithilfe der DP erschließt sich jedoch ein iterativer Weg.
\par 
Da die Algorithmen der Dynamischen Programmierung Zugriff auf die Übergangswahrscheinlichkeiten der Umwelt sowie der Belohungsfunktion haben müssen, ist ein perfektes Modell der Umgebung Vorraussetzung. DP-Methoden sind folgerichtig immer \textit{model-based}.
\par

\subsubsection{Strategieevaluierung}
Wichtiger Bestandteil eines jeden RL Algorithmus ist die Berechnung der Zustands-Nutzenfunktion $v_\pi$ (oder Aktions-Nutzenfunktion $q_\pi$) für eine willkürliche Strategie $\pi$. Die geschätzte Nutzenfunktion (unter einer bestimmten Strategie) gegen die wahren Werte der Gewinne streben zu lassen, wird auch als Strategieevaluierung (\textit{Policy Evaluation}) bezeichnet \cite[S.~74]{Sutton1998}.
\par 
Die Vorgehensweise der DP um dieses Vorhersageproblem (\textit{Prediction Problem}) zu lösen, lässt sich folgendermaßen beschreiben. Zunächst wird eine willkürliche Nutzenfunktion $v_0$ gewählt, z.B. sind alle Nutzen zu Beginn mit 0 definiert. Die Bellman-Gleichung wird nun als Aktualisierungsregel angesehen, die Schritt für Schritt die geschätzten Nutzen verbessert. Es entsteht eine Reihenfolge von geschätzten Nutzenfunktionen $v_0$, $v_1$, $v_2$, \dots, die letztendlich zu $v_\pi$ konvergiert. \cite{Sutton1998} definieren die Aktualisierungsregel wie folgt (S.~74):
\begin{equation}\label{eq:fullbackup}
    \begin{aligned}
        \forall s \in \mathcal{S}: v_{k+1}(s) &= \EX_{\pi}[ R_{t+1} + \gamma v_k(S_{t+1}) \mid S_t = s] \\
        &= \sum_a{\pi(a|s) \sum_{s',r}p(s',r\mid s, a) \left[r+\gamma v_k(s')\right]}
    \end{aligned}
\end{equation}

Mit Worten beschrieben, wird die Aktualisierungsregel in jeder Iteration auf alle Zustände $s \in \mathcal{S}$ angewendet. Dabei wird der alte Nutzen eines Zustands durch einen neuen Nutzen ersetzt, der auf dem erwarteten Nutzen aller Nachfolgezustände und der sofortigen Belohnung beruht, gewichtet nach den Übergangswahrscheinlichkeiten \cite[S.~20]{Wiering}. Hierbei ist festzustellen, dass die  Aktualisierung des geschätzen Nutzen für einen Zustand auf den ebenfalls geschätzen Nutzen der nachfolgenden Zustände stattfindet. DP-Methoden benutzen also das Prinzip des \textit{bootstrapping} \cite[S.~89]{Sutton1998}.

\subsubsection{Strategieverbesserung}
Ist eine suboptimale Strategie vollständig evaluiert worden, d.h. ist die Nutzenfunktion $v_\pi$ präsent, dann kann diese genutzt werden, um eine besser Strategie zu finden. Dazu wird zunächst $q_\pi$ berechnet durch \cite[S.~21]{Wiering}:

\begin{equation}\label{eq:qBerechnung}
    \begin{aligned}
        q_\pi(s,a) = \EX_\pi\left[ R_{t+1} + \gamma v_\pi(S_{t+1}) \mid S_t = s, A_t=a\right]
    \end{aligned}
\end{equation}

Für den Fall, dass $q_\pi(s,a)$ größer ist als $v_\pi$ für ein $a \in \mathcal{A}$, dann ist es besser die Aktion $a$ auszuführen, als jene Aktion, die durch die aktuelle Strategie $\pi$ gewählt wird. Wird diese Verbesserung (\textit{Policy Improvement}) für alle Zustände durchgeführt, dann ergibt sich die gierige (\textit{greedy}) Strategie $\pi'$, die die besten Aktionen basierend auf die aktuelle Nutzenfunktion ausführt.

\subsubsection{Strategieiteration}
Eine Methode, die die zwei Prozesse zum Evaluieren und der Verbesserung einer Strategie zusammenführt, ist die sog. Strategieiteration \textit{Policy Iteration}, die ihren Ursprung in den Arbeiten von \cite{bellman1957dynamic} und \cite{howard1960dynamic} hat. 
\par 
Bei der Strategieiteration wird zunächst eine willkürliche Strategie $\pi_0$ gewählt, die anschließend zu der Nutzenfunktionen $v_{\pi_k}$ evaluiert wird. Ist dieser Schritt abgeschlossen, wird die Strategie gemäß der berechneten Aktions-Nutzenfunktion $q_{\pi_k}$ (vgl. \ref{eq:qBerechnung}) verbessert, aus $\pi_k$ folgt $\pi_{k+1}$ und eine erneute Evaluation kann erfolgen. Die Schleife wird gestoppt, wenn für alle Zustände $s$ gilt, dass $\pi_{k+1}(s) = \pi_k(s)$ \cite[S.~22]{Wiering}. 
\par 
Die Strategieiteration generiert somit folgende Sequenz \cite[S.~22]{Wiering}:
\begin{equation}\label{eq:policyItSeq}
\pi_0 \rightarrow v_{k_0} \rightarrow  \pi_1 \rightarrow v_{k_1} \rightarrow  \pi_2 \rightarrow v_{k_2}\rightarrow  \pi_3 \rightarrow v_{k_3}\rightarrow \dots \rightarrow \pi_*
\end{equation}

\subsubsection{Nutzeniteration}
In Kapitel 3.1.1 wurde gezeigt, dass die Evaluierung einer Strategie mehrere Iteration durchlaufen muss, um letzendlich zu $v_\pi$ zu konvergieren. Es ist jedoch möglich diesen Prozess vorzeitig abzubrechen, \glqq ohne die Garantie der Konvergenz der Stratgieiteration zu verlieren\grqq{} \cite[S.~82]{Sutton1998}. 
\par 
Diese Erkenntnis macht sich die sog. Nutzeniteration (\textit{Value Iteration}) zu nutzen, die bereits nach einer Iteration der Evaluation abbricht und den Schritt zur Verbesserung der Strategie direkt im Bezug auf diese Berechnung ausführt. Die Nutzeniteration fokusiert sich somit ausschließlich auf Schätzung der Nutzenfunktion und erzeugt im Vergleich zur Strategieiteration folgende Sequenz \cite[S.~23]{Wiering}:

\begin{equation}\label{eq:valueItSeq}
    v_0 \rightarrow v_1 \rightarrow v_2 \rightarrow v_3 \rightarrow \dots \rightarrow v_*
\end{equation}


	\subsection{Monte-Carlo Methoden}\label{sec:MC}
	Dieses Kapitel beschäftigt sich mit der Gruppe der Monte-Carlo Lernmethoden. Zunächst wird die Grundidee dieses \textit{model-free} Ansatzes erläutert, d.h. wie die MC-Methoden das Vorhersageproblem lösen. Anschließend wird darauf eingegangen, wie mit dem Exploration-Exploitation Dilemma umgegangen wird, gefolgt von der Darstellung des Pseudocodes für den \textit{first-visit}-Algorithmus.
\par 

Im vorrigen Kapitel wurde die Dynamische Programmierung vorgestellt. Ein Verfahren, welches ein komplettes Modell der Umgebung benötigt und somit als \textit{model-based} bezeichnet wird. Die Hauptdisziplin des Reinforcement Learning ist jedoch die Suche nach optimalem Verhalten, wenn kein Zugriff auf die Dynamiken der Umwelt vorhanden ist \cite[S.~27]{Wiering}. Diese \textit{model-free} Methoden lernen und approximieren aufgrund der Erfahrung, die sie durch die Interaktion mit der Umwelt erwerben. Dabei kann entweder mit der tatsächlichen Umwelt interagiert werden oder mit einer Simulation.
\par 
Im Bezug auf Simulationen erwähnen \cite{Sutton1998} eine interessante Aussage. Zwar müsse ein Modell der Umwelt für eine Simulation vorhanden sein, aber sie behaupten, dass es in überraschend vielen Fällen möglich sei, Erfahrung aufgrund der erwünschten Wahrscheinlichkeitsverteilung zu erzeugen ohne dabei die komplette Wahrscheinlichkeitsverteilung für alle möglichen Übergänge zu kennen wie sie z.B. bei der DP benötigt wird (S.~91).
\par 
// TODO: Lindemann; Beispiel? Würfel, BlackJack? Anstatt die Wahrscheinlichkeiten immer wieder komplett auszurechnen, kann man einfach einen Kartenstapel simulieren, von dem Karten entfernt werden. Meinen die beiden das?
\par 

\subsubsection{Vorhersageproblem}
Monte-Carlo Methoden lösen das Reinforcement Learning Problem über Durschnittsbildung der, durch Erfahrung gesammelten, Gewinne \cite[S.~91]{Sutton1998}. Dabei werden die Nutzen und die Strategien ausschließlich nach einer abgeschlossenen Episode aktualisiert. Folgerichtig ist das Aktualisierungsverhalten \textit{episode-by-episode} und nicht \textit{step-by-step} (als nach jeder Aktion)\cite[S.~91]{Sutton1998}. Das ist zugleich der Hauptunterschied zu den in Kapitel 4 vorgestelltem Temporal-Difference Learning, welches nach jeder Aktion die geschätzen Nutzen anpasst. Daraus folgt auch, dass Monte-Carlo Methoden ausschließlich auf episodiale Probleme anwendbar sind, TD Learning jedoch zusätzlich auch bei kontinuierlichen Aufgaben zum Einsatz kommen kann. 
\par
Zur Erinnerung, eine Nutzenfunktion gibt den Nutzen eines Zustands an, also den geschätzten erwartbaren Gewinn. Dabei ist der erwartete Gewinn eines Zustands die erwartbare Summe aller zukünftig, diskontierten Belohnungen, wenn von diesem Zustand aus gestartet wird. Um den erwarteten Gewinn zu schätzen, kann der Durschnitt über die, durch Erfahrung gesammelten, realen Gewinne gebildet werden.
\par 
Der Grundansatz ist dabei wie folgt. Eine Episode unter der Strategie $\pi$ wird z.B. durch eine Simulation erzeugt. Es entsteht eine Reihenfolge von Triple $(s,a,r)$. Kommt ein Zustand $s$ innerhalb der Episode vor, wird auch von einem Besuch (\textit{visit}) von $s$ gesprochen (im Rahmen der Monte-Carlo Methoden). Der Gewinn für den Zustand $s$ ist somit die Summe aller Belohnung nach dem ersten Besuch des Zustands $s$. Über alle, auf diese Weise gesammelten, Gewinne von $s$ wird der Durschnitt berechnet, der -mit steigender Anzahl an Besuchen- gegen den tatsächlichen Nutzen von $s$ strebt. Wird der Gewinn vom Start des ersten Besuchs von $s$ berechnet, dann wird diese Methode auch als \textit{First-Visit} bezeichnet. Konsequenterweise bezeichnet \textit{Every-Visit} die Methode, bei der für jegliche Besuche von $s$ in einer Episode der Gewinn berechnet wird. Diese beiden Methoden sind sehr ähnlich und unterscheiden sich z.B. im Pseudocode nur durch eine Abfrage. Trotzdem haben sie unterschiedliche theoretische Eigenschaften \cite[S.~92]{Sutton1998}. Diese sind jedoch im Rahmen dieser Arbeit nicht weiter dargestellt, da ausschließlich mit der \textit{First-Visit} Variante garbeitet wird. //TODO für mehr Infos
\par 
In Kapitel 2.6 ist erwähnt worden, dass \textit{model-free} Lernmethoden die Aktions-Nutzenfunktion berechnen. Grund hierfür ist, dass sie nicht in der Lage sind einen Schritt vorherzusehen, weil die Übergangsfunktion $p$ nicht gegeben ist. Monte-Carlo Methoden können sowohl, wie im vorrigen Absatz erläutert, den Nutzen von Zuständen berechnen, als auch den Nutzen von Zustands-Aktions-Paaren. Der Unterschied besteht darin, dass nicht der Besuch von $s$ entscheidend ist, sondern der Besuch des Paares $(s,a)$.

\subsubsection{Exploration}\label{sec:exploration}
Da die Monte-Carlo Methoden mittels Durchschnittsbildung arbeiten, ist es unabdingbar, dass jeglichen Zustände respektive Zustands-Aktions-Paare ausreichend oft besucht werden. Wenn jedoch eine deterministische Strategie $\pi$ gegeben ist, dann wird immer nur eine Aktion pro Zustand ausgeführt, nämlich jene mit dem derzeitigen höchsten, geschätzten Nutzen. Dies sorgt dafür, dass der Zustands- und Aktionsraum nicht ausreichend erkundet wird und der Algorithm sozusagen in einem lokalem Maximum festhängt.
\par 
\cite{Sutton1998} stellen in ihrem Werk drei Ansatze vor, die die fortlaufende Exploration ermöglichen. Eine Möglichkeit ist die Verwendung einer $\epsilon$\textit{-greedy} Strategie, wie sie auch im Kapitel 2.8 vorgestellt wurde. Mit einer bestimmten Wahrscheinlichkeit $\epsilon$ wird nicht die vermeintlich beste Aktion gewählt (aktuell größter  Aktions-Nutzen), sondern eine zufällige Aktion $a \in \mathcal{A}$. Diese Methodik erlaubt die fortlaufende Exploration und garantiert trotzdem eine Konvergenz zu einer optimalen Strategie, wenn $\epsilon$ im Laufe der Zeit verringert wird \cite[S.~201]{Sutton1998}. Welche Auswirkungen die Werte von $\epsilon$ auf das Konvergenzverhalten haben wird im Rahmen der praktischen Umsetzung anhand des JumpingDino Beispiels in Kapitel \ref{sec:resJumpSimple} untersucht.
\par 
Um einen einen Überblick über weitere Vorgehensweisen zu der fortlaufenden Erkunden zu geben, werden auch die zwei weiteren Methoden nach \cite{Sutton1998} kurz dargestellt (S.96-108). 
\par 
Anstatt die Strategie so zu verändern, dass sie suboptimale Aktionen wählt, um alle Zustände oder Zustands-Aktions-Paare ausreichend oft zu besuchen, kann auch explizit mit einem bestimmten Zustand respektive Zustands-Aktions-Paar gestartet werden. Dabei muss jeder Zustand oder jedes Zustands-Aktions-Paar eine Wahrscheinlichkeit größer 0 haben, um als Start einer Episode ausgewählt zu werden. Dieser Ansatz wird auch als \textit{Exloring Starts} bezeichnet.
\par
Eine weitere Möglichkeit ist das sog. \textit{off-policy learning}. Die Grundidee hierbei besteht darin, nicht eine Strategie zu benutzen, die teilweise exploriert ($\epsilon$\textit{-greedy}), sondern zwei Strategien zu verwenden. Eine konvergiert zu der optimalen Strategie und die andere exploriert den Zustands- und Aktionsraum, sammelt somit die Erfahrung. Die Strategie, die stetig verbessert wird, wird als Zielstrategie (\textit{target policy}) bezeichnet wohingegen die Strategie, die die Episoden erzeugt als Verhaltensstrategie (\textit{behavior policy}) bezeichnet wird \cite[S.~103]{Sutton1998}. Das \glqq off \grqq{} in \textit{off-policy learning} bezieht sich darauf, dass die Erfahrung einer anderen, von der Zielstrategie abweichenden, Strategie dazu genutzt wird, um zu lernen.
\pagebreak
\subsubsection{Pseudocode}
Der nachfolgende Pseudocode \cite[S.~101]{Sutton1998} zeigt die Vorgehensweise der Monte-Carlo Methoden zur Bestimmung der optimalen Strategie $\pi_*$ auf Basis der Aktions-Nutzenfunktion $q$ bzw. $Q$. Genauer wird die \textit{First-Visit} Variante vorgestellt, die mithilfe einer $\epsilon$-greedy Strategie, die die fortlaufende Erkunden garantiert. Wie in Kapitel 2.4 erläutert, sollte der Diskontierungsfaktor $\gamma$ für episodiale Probleme den Wert 1 annehmen, um jegliches Handeln während einer Episode bei der Berechnung des Gewinns zu berücksichtigen.
\par 
\begin{algorithm}
    \caption{On-policy first-visit MC control (for $\epsilon$-soft policies), estimates $\pi \approx \pi_*$}
    \begin{algorithmic}[1]
        \State Algorithm parameter: small $\epsilon > 0$
        \State Initialize:
        \Indent
           \State $\pi \gets$ an arbitary $\epsilon$-soft policy
           \State $Q(s,a) \in \mathbb{R}$ (arbitrarily), for all $s \in \mathcal{S}, a \in \mathcal{A}(s)$
           \State $Returns(s,a) \gets$ empty list, for all $s \in \mathcal{S}, a \in \mathcal{A}(s)$
        \EndIndent
        \State Repeat forever (for each episode):
        \Indent
            \State Generate an episode following $\pi: S_0, A_0, R_1, \dots, S_{T-1}, A_{T-1}, R_T$
            \State $G \gets 0$
            \State Loop for each step of episode, $t= T-1,T-2, \dots, 0:$
            \Indent
                \State $G \gets \gamma G + R_{t+1}$
                \State Unless the pair $S_t, A_t$ appears in $S_0, A_0, S_1, A_1, \dots ,S_{t-1}, A_{t-1}:$
                \Indent
                    \State Append $G$ to $Returns(S_t,A_t)$
                    \State $Q(S_t,A_t) \gets$ average$(Returns(S_t,A_t))$
                    \State $A^* \gets \argmax_a Q(S_t, a)$ (with ties broken arbitrarily)
                    \State For all $a \in \mathcal{A}(S_t):$
                    \Indent
                     \State  $\pi(a|S_t) =   
                        \begin{cases}
                            1-\epsilon + \epsilon / |\mathcal{A}(S_t)|      & \quad \text{if } a = A^* \\
                            \epsilon / |\mathcal{A}(S_t)|  & \quad \text{if } a \neq A^*
                        \end{cases}$
                    \EndIndent
                \EndIndent
            \EndIndent
        \EndIndent 
    \end{algorithmic}
\end{algorithm}
Eine Umformung zur \textit{Every-Visit} Variante kann durch das Entfernen der Bedingung in Zeile 11 realisiert werden.
\par 
Dieser Ansatz der Monte-Carlo Methode folgt dem Schema der \textit{Generalized Policy Iteration}, vgl. Kapitel \ref{sec:GPI}. Der Prozess der Strategieevaluation findet nach jeder Episode statt und benötigt im Vergleich zu der Dynamischen Programmierung kein perfektes Modell. Da die Evaluation für jeden Zustand bzw. jedes Zustands-Aktions-Paar nach nur einem Schritt (der Durschnittsermittlung des Gewinns) gestoppt wird und danach der Prozess der Strategieverbesserung (\textit{Policy Improvement}) startet, errinert dieses Vorgehen an die Nutzeniteration aus der Dynamischen Programmierung, siehe Kapitel \ref{sec:Nutzeniteration}.  Berechnete Werte für den Aktions-Nutzen jedes Zustands dienen als Grundlage für den Prozess der Strategieverbesserung, dem Verbessern der Strategie, im Fall der Monte-Carlo Methoden, nach jeder Episode.
\subsubsection{BlackJack Beispiel*}

\subsubsection{Zusammenfassung}
Monte-Carlo Methoden lernen Nutzenfunktionen durch die direkte Interaktion mit der Umgebung. Damit zählen sie zu den \textit{model-free} Lernmethoden, die kein perfektes Modell der Umgebung benötigen. Zur Ermittlung des erwarteten Gewinns für ein Zustands-Aktions-Paars wird der Durschnitt über jegliche erhaltene Gewinne pro Episode gebildet. Somit findet die Evaluation und Verbesserung einer Strategie immer nur nach dem Abschluss einer Episode statt. Dies ist zugleich der Grund, warum Monte-Carlo Methoden ausschließlich auf episodiale Probleme anwendbar sind.
\par 
Da Aktionen auf Basis der temporär besten Aktions-Nutzen gewählt werden, ist eine ausreichende Exploration nicht gegeben, weil Gewinne vermeintlich suboptimaler Zustands-Aktions-Paare nicht gesammelt werden. Der Algorithmus verharrt in einem lokalem Maximum. Um dies zu verhindern, kann eine $\epsilon$-greedy Strategie verwendet werden, die mit einer Wahrscheinlichkeit von $\epsilon$ eine zufällige Aktion ausführt.
\par 
Im Vergleich zu der Dynamischen Programmierung benötigen die MC Methoden kein perfektes Modell der Umgebung und können auf Basis von Simulationen lernen. Zugleich aktualiseren sie ihre geschätzten Nutzen nicht auf Basis von anderen geschätzten Nutzen, sie betreiben somit kein \textit{bootstrapping}. 
\par 
Im nächsten Kapitel werden Lernmethoden vorgestellt, die wie die MC Methoden kein perfektes Modell benötigt, aber wie die DP \textit{bootstrappen} und somit in der Lage sind, nach jedem Zeitstempel ihre geschätzten Nutzen zu aktualiseren.
	\subsection{Temporal Difference Learning}\label{sec:TD}
	
Nachdem in den beiden vorrigen Kapitel die Methoden der Dynamischen Programmierung und des Monte-Carlo Ansatzes beleuchtet wurden, befasst sich dieses Kapitel mit der dritten großen Gruppe an Algorithmen, die das Reinforcement Learning Problem lösen, dem \textit{Temporal-Difference Learning} (TD).
Wie zuvor wird zunächst erläutert, wie diese Art der Algorithmen das Vorhersageproblem lösen. Anschließend werden zwei vollständige Algorithmen vorgestellt, die das Kontrollproblem, also die Suche nach einer optimalen Strategie, bewältigen.
\par 
Um einschätzen zu können, welche zentrale Rolle das TD in dem Bereich des Reinforcement Learnings eingenommen hat, folgt ein Zitat von \cite{Sutton1998}:
\begin{quote}
    If one had to identify one idea as central and novel to reinforcement learning, it would undoubtedly be temporal-di↵erence (TD) learning. TD learning is a combination of Monte Carlo ideas and dynamic programming (DP) ideas. \cite[S.~119]{Sutton1998}
\end{quote}

Die Verbindung besteht zum einen daraus, dass das TD genau wie die Monte-Carlo Methoden direkt durch die Interaktion mit der Umwelt lernt, folgerichtig auch \textit{model-free} sind. Zum anderen aktualisieren die TD Methoden, genauso wie bei der Dynamische Programmierung, ihre geschätzen Nutzen mit Hilfe weiterer geschätzen Nutzen, sie bedienen sich also ebenfalls dem Konzept des \textit{bootstrapping} \cite[S.~119]{Sutton1998}. Dadurch ist das TD in der Lage, seine Nutzentabelle nach jeder Aktion zu aktualisieren, \textit{step-by-step}. Ein Warten auf das Ende einer Episode, wie bei den MC-Methoden, ist nicht notwendig. TD kann somit zusätzlich bei kontinuierlichen Problem zum Einsatz kommen \cite[S.~124]{Sutton1998}.
\par 

\subsubsection{Vorhersageproblem}
MC- und TD-Methoden lösen das Vorhersageproblem beide durch gesammelte Erfahrung durch die direkte Interaktion mit der Umwelt. Sie folgen einer Strategie $\pi$ und aktualisieren ihre geschätzen Nutzen $V$ (oder $Q$) für $v_\pi$ (respektive $q_\pi$) auf Grundlage der erhaltenen $(s,a,r)$ Triple. Doch wie schafft es das \textit{Temporal-Differene Learning} im Gegensatz zu den Monte Carlo Methoden nach jedem Schritt zu aktualisieren und nicht auf das Ende einer Episode zu warten, somit nicht den Gewinn $G_t$ zu benötigen?
\par 
Um diese Frage zu beantworten, wird zunächst ein neuer Parameter vorgestellt, die Schrittgröße $\alpha$ (\textit{step-size parameter}). Dieser Parameter beeinflusst die Lernrate und sorgt konkret dafür, wie stark die Veränderung eines neu geschätzen Nutzens gewichtet wird. Des Weiteren wird der Begriff \glqq Ziel\grqq{} (\textit{target}), im Umfeld von TD auch TD-Ziel (\textit{TD-target}), verwendet. Dieses Ziel sagt aus, zu welchem Wert die derzeitige Aktualiserung des Nutzen strebt.
\par 
Monte-Carlo Methoden müssen bis zu dem Ende einer Episode warten, da erst dann der Gewinn $G_t$ feststeht, der als Ziel für $V(S_t)$ benötigt wird \cite[S.~119]{Sutton1998}. Eine vereinfachte formale Darstellung der Aktualisierungsregel für die Ever-Visit MC-Methode sieht wie folgt aus \cite[S.~119]{Sutton1998}:

\begin{equation}
    V(S_t) \leftarrow V(S_t) + \alpha \left[G_t - V(S_t)\right]
\end{equation}

Im Gegensatz dazu, müssen die TD-Methoden lediglich bis zu dem nächsten Zeitstempel warten, um eine Aktualiserung vorzunehmen. Dazu wird zum Zeitpunkt $t+1$ sofort ein Ziel gebildet, welches aus der Belohnung $R_{t+1}$ und dem geschätzen Nutzen $V(S_{t+1})$ zusammengesetzt ist. Die Aktualisierungsregel für die einfachste Form des TD lautet somit \cite[S.~120]{Sutton1998}:
\begin{equation}
    V(S_t) \leftarrow V(S_t) + \alpha \left[R_{t+1} + \gamma V(S_{t+1}) - V(S_t)\right]
\end{equation}
Statt dem Ziel $G_t$ der MC-Methoden, ist das Ziel des TD-Learnings $R_{t+1} + \gamma V(S_{t+1})$. Da der Wert für $V(S_{t+1})$ ein geschätzer Wert ist, aber dennoch für die Aktualiserung verwendet wird, \textit{bootstrappt} das TD-Learning. Dies ist notwendig, um nicht den realen Gewinn nach Abschluss einer Epsiode verwenden zu müssen, sondern diesen gewissermaßen aufspalten zu können und nur auf Basis der aktuellen Belohnung eine Anpassung vorzunehmen. Diese Aufspaltung basiert auf der fundamentalen Erkenntnis, dass Gewinne aufeinanderfolgender Zeitstempel in Verbindung stehen (siehe Kapitel \ref{Gewinne}) und somit gilt \cite[S.~120]{Sutton1998}: 
\begin{equation}\label{eq:targets}
\begin{aligned}
v_\pi &= \EX_\pi\left[G_t \mid S_t = s \right] \\
&= \EX_\pi\left[R_{t+1} + \gamma G_{t+1} \mid S_t = s \right] \\
        &= \EX_\pi\left[R_{t+1} + \gamma v_\pi(S_{t+1}) \mid S_t = s \right]
\end{aligned}
\end{equation}

Anhand von \ref{eq:targets} lässt sich der Zusammenhang der drei großen Gruppen von Lernmethoden sehr gut zusammenfassen. Die erste Zeile beschreibt den geschätzen Wert, den die Monte-Carlo Methoden als Ziel verwenden. Es handelt sich um einen Schätzwert, da der Erwartungswert unbekannt ist und stattdessen mit dem Durschnitt gesammelter Gewinne gerechnet wird. 
\par 
Die Dynamische Programmierung benutzt den Schätzwert, der sich aus der dritten Zeile ergibt. Dabei bezieht sich das Schätzen nicht auf die eigentlichen Erwartungswerte, denn diese können berechnet werden, da ein perfektes Modell der Umgebung mit allen Übergangswahrscheinlichkeiten vorhanden ist. Ausschlaggebend ist, dass $v_\pi(S_{t+1})$ zum Zeitpunkt $t$ nicht berechnet wird, sondern von dem derzeitige geschätzte Nutzen $V_{t+1}$ Gebrauch gemacht wird \cite[S.~120]{Sutton1998}.
\par 
Das TD-target ist eine Schätzung aufgrund beider Gründe, die in den zwei vorrigen Absätzen erläutert wurden. Es basiert auf der Sammlung von Erfahrung, um den Erwartungswert bzw. die Werte in der dritten Zeile von \ref{eq:targets} schätzen zu können und gleichzeitig wird der derzeitg geschätze Nutzen $V$ anstelle des wahren Wertes von $v_\pi$ verwendet \cite[S.~120f]{Sutton1998}.
\par
\cite{Sutton1998} fassen sehr gut zusammen: \glqq TD vereinigt die Probenahme (\textit{sampling}) der MC-Methoden mit dem \textit{bootstrapping} der DP\grqq{} (S.~121).

\pagebreak
\subsubsection{SARSA}
\begin{algorithm}
    \caption{Sarsa (on-policy TD control) for estimating $Q \approx q_*$}
    \begin{algorithmic}[1]
        \State Algorithm parameter: step size $\alpha \in (0,1])$, small $\epsilon > 0$
        \State Initialize $Q(s,a),$ for all $s \in S^+, a \in \mathcal{A}(s),$ arbitrarily except that \\ $Q(terminal, \mathord{\cdot}) = 0$
        \\
        \State Loop for each episode:
        \Indent
            \State Initialize $S$
            \State Choose $A$ from $S$ using policy derived from $Q$ (e.g., $\epsilon$-greedy)
            \State Loop for each step of episode:
            \Indent
                \State Take action $A$, observe $R, S'$
                \State Choose $A'$ from $S'$ using policy derived from $Q$ (e.g., $\epsilon$-greedy)
                \State $Q(S,A) \gets Q(S,A) + \alpha [R + \gamma Q(S',A') - Q(S,A)]$
                \State $S \gets S'; A \gets A';$
            \EndIndent
            \State until $S$ is terminal
        \EndIndent 
    \end{algorithmic}
\end{algorithm}


\pagebreak
\subsubsection{Q-Learning}
\begin{algorithm}
    \caption{Q-Learning (off-policy TD control) for estimating $\pi \approx \pi_*$}
    \begin{algorithmic}[1]
        \State Algorithm parameter: step size $\alpha \in (0,1])$, small $\epsilon > 0$
        \State Initialize $Q(s,a),$ for all $s \in S^+, a \in \mathcal{A}(s),$ arbitrarily except that \\ $Q(terminal, \mathord{\cdot}) = 0$
        \\
        \State Loop for each episode:
        \Indent
            \State Initialize $S$
            \State Loop for each step of episode:
            \Indent
                \State Choose $A$ from $S$ using policy derived from $Q$ (e.g., $\epsilon$-greedy)
                \State Take action $A$, observe $R, S'$
                \State $Q(S,A) \gets Q(S,A) + \alpha [R + \gamma \max_a Q(S',a) - Q(S,A)]$
                \State $S \gets S';$
            \EndIndent
            \State until $S$ is terminal
        \EndIndent 
    \end{algorithmic}
\end{algorithm}

\subsubsection{Zusammenfassung}


\section{Praktischer Teil}
	\subsection{Implementierung}
	

\subsubsection{Anforderungen}
Zu Beginn und während der Umsetzung kristallisierten sich Anforderungen heraus, die zunächst aufgezählt und anschließend kurz erläutert werden:

\begin{itemize}
    \item Gut gewählte Interfaces, die die Theorie widerspiegeln
    \item Determinismus; Wiederholbare Ergebnisse
    \item Visualisierung; GUI
    \item Erweiterbarkeit
    \item Sammlung von Statistiken
    \item Lernprozess speichern
\end{itemize}

\textit{Gut gewählte Interfaces}. Die Interfaces sind so geschnitten, dass sie die grundlegenden Bestandteile des Reinforcement Learnings widerspiegeln. Dabei richtet sich die Terminologie an das Agent-Umwelt Interface, welches in Kapitel 2.1 vorgestellt ist. 
\par 
\textit{Determinismus}. Um zu gewährleisten, dass gesammelte Ergebnisse zum Verhalten von unterschiedlichen Algorithmen vergleichbar sind, muss die gesammte Implementierung determistisch sein und wiederholbare Ergebnisse liefern. Eines der wichtigsten Faktoren hierbei ist die Handhabung des \textit{Random Number Generators}, der vor allem dafür benötigt wird, um \glqq willkürliche\grqq{} Aktionen bei $\epsilon$\textit{-greedy} Strategien zu wählen. Gearbeitet wird ausschließlich mit der \textit{RNG.java} Klasse, die mit Hilfe eines \textit{static}-Kontruktors einmalig ein \textit{Random}-Objekt anlegt und \textit{seeded}.
\par 
Eine wichtige Erkenntnis ist außerdem, dass die HashMap in Java nicht deterministisch agiert im Bezug auf die Reihenfolge der Elemente. Die Reihenfolge ist jedoch entscheidend, da u.a. eine HashMap die Aktionen auf ihre Nutzen abbildet und das \textit{KeySet} dieser Map dazu benutzt wird, um eine willkürliche Aktion zu wählen. In der Dokumentation zu der \textit{HashMap}-Klasse heißt es: \glqq This class makes no guarantees as to the order of the map; in particular, it does not guarantee that the order will remain constant over time \grqq{} []\cite{hashmap}.
Um dennoch eine feste Reihenfolge zu garantieren, wird ausschließlich die \textit{LinkedHashMap} als Implementationen des \textit{Map}-Interfaces benutzt. Diese sichert eine konsistente, vorhersagebare Ordnung zu \cite[]{linkedHashMap}.

Die umgesetzten Algorithmen sind alle iterativ und somit \textit{single-threaded}. Dennoch werden weitere nebenläufige Threads eingesetzt, um z.B. Laufzeitstatistiken zu sammeln. Auch das UI läuft in einem seperaten Thread. Es muss somit darauf geachtet werden, nur geeignete Aufgaben in andere Threads auszulagern, die den eigentlichen Lernprozess im Main-Thread nicht beeinflussen.
\par 
\textit{Visualisierung}. Wenn über tausende Episoden gelernt wird, Millionen von Belohnungen verteilt und Aktionen ausgeführt werden, dann reicht eine simple Konsolenausgabe nicht mehr aus, um das Verhalten eines Algorithmus einzuschätzen. Eine Graphische Nutzungsoberfläche (\textit{GUI}) ist erstellt worden, mit der Parameter während des Lernens gesteuert werden können. Außerdem wird die Umwelt visualisiert, die Nutzentabelle kann angezeigt werden und ein kontinuierlicher Graph zeigt die erhaltenen Belohnungen an.

\textit{Erweiterbarkeit}. Der Aufbau der Implementierung erlaubt ein einfaches Hinzufügen von weiteren Lernszenarien. Hierzu muss lediglich ein \textit{Enum}, welches den Aktionsraum repräsentiert und eine Klasse, die das \textit{Environment}-Interface implementiert, anlegt werden. Ebenfalls können weitere RL Algorithmen ergänzt werden, indem von der abstrakten Klasse \textit{Learning} bzw. \textit{EpisodicLearning} abgeleitet wird. Letztendlich ergibt sich eine Art RL-Framework, welches unterschiedliche Umgebungen und Algorithmen bereitstellt.

\textit{Sammlung von Statistiken}. Um z.B. Aussagen über das Konvergenzverhalten von unterschiedlichen Lernmethoden treffen zu können, ist es notwendig, Daten zu sammeln und auszuwerten. Das umgesetzte \textit{Listener}-Pattern, bei dem die Algorithmen u.a. nach jedem Zeitstempel oder nach jeder kompletten Episode ein Event mit Informationen auslösen, erlaubt eine generische Datenerhebung für unterschiedliche Lernmethoden und Problemstellungen.

\textit{Lernprozess speichern}. Alle Methoden des Reinforcement Learnings, die im Rahmen dieser Arbeit implementiert sind, laufen \textit{single-threaded} und benötigen bei großen Zustands- und Aktionsräumen (100.000+ Zustände) lange Laufzeiten. Daher ist eine Speichern- und Laden-Funktion umgesetzt, die die aktuelle Nutzentabelle serialisieren und deserialisieren kann, um einen Lernvorgang zu einem späteren Zeitpunkt fortzusetzen.

\subsubsection{Interfaces}
\begin{figure}[H]
    \centering
    \includegraphics[width=0.7\textwidth]{images/Interfaces.png}
    \caption{Darstellung der wichtigsten Interfaces}
    \label{fig:GPI}
\end{figure}
	\subsection{Jumping Dino}
		\subsubsection{Problemstellung}
		\subsubsection{Zustandsmodellierung}
		\subsubsection{Konvergenzverhalten}
	\subsection{Ant-Game}
		\subsubsection{Problemstellung}
		\subsubsection{Zustandsmodellierung}
		\subsubsection{Konvergenzverhalten}

\section{Fazit}
\section{Ausblick}

\pagebreak
\bibliography{quellen}
\pagebreak

Bellman Optimality Equation:
\begin{equation}\label{eq:bellman}
    \begin{aligned}
    q_*(s,a) &= \EX [R_{t+1} + \gamma \max_{a'}q_*(S_{t+1}, a') \mid S_t = s, A_t = a] \\
    &= \sum_{s',r}p(s', r\mid s, a)[r + \gamma \max_{a'}q_*(s', a')]
    \end{aligned}
\end{equation}
\end{document}
