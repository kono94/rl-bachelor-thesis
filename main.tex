\documentclass[12pt]{article}
\usepackage{comment} % enables the use of multi-line comments (\ifx \fi) 
\usepackage[english,german]{babel} 
\usepackage[utf8]{inputenc}
\usepackage{fancyhdr}
\usepackage{graphicx}
\usepackage[article, total={6in, 8in}]{geometry}
\usepackage{float}
\usepackage{wrapfig}
%\usepackage[onehalfspacing]{setspace}
\usepackage{url}
\usepackage{tabularx}
\usepackage{pdfpages}
\usepackage{wrapfig}
\usepackage[T1]{fontenc}% wichtig für Trennung von Wörtern mit Umlauten
\usepackage{microtype}% verbesserter Randausgleich
\setlength{\headheight}{21.4pt}
\usepackage{apacite}
\usepackage{eurosym}
\usepackage{amsmath}
\numberwithin{equation}{section}
% Special characters
\usepackage{eucal}
% math. alphabet
\usepackage{amssymb}
\usepackage{scrpage2}%
\DeclareMathOperator*{\argmax}{argmax}

%%%% footer and header %%%%
\usepackage{scrpage2}%
\pagestyle{scrheadings}%  S
\clearscrheadfoot% 
\setheadwidth{text}%
\automark{section}% 
\ihead{\textbf{\pagemark}}
\renewcommand{\sectionmark}[1]{\markright{\ #1}} 
\ohead{\rightmark}
\setheadsepline{0.5pt}
%%%% \footer and header %%%%

\begin{document}

\begin{titlepage}
	\centering	
	{\scshape\LARGE Hochschule Bremerhaven \par}
	\vspace{1cm}
	{\scshape\Large Exposé für eine Bachelorarbeit zum Thema:\par}
	\vspace{1.5cm}
	{\huge\bfseries Reinforcement Learning\par}
	\vspace{2cm}
	{\Large\itshape Theoretische Grundlagen der tabellarischen Lernmethoden und praktische Umsetzung am Beispiel eines Ameisen-Agentenspiels
	\par}
	\vfill
	\begin{tabularx}{\textwidth}{lX}
		Autor: & Jan Löwenstrom \\
		Matrikelnr.: & 34937 \\
		Erstprüfer: & Prof. Dr.-Ing. Henrik Lipskoch \\
		Zweitprüfer: & Prof. Dr. Nadija Syrjakow \\
	\end{tabularx}  
    \vfill

% Bottom of the page with current date
	{\large \today \par}       
\end{titlepage}

\tableofcontents
\pagebreak
\listoffigures
\newpage


\noindent
\section{Einleitung}

Das ist eine Einleitung8us
\pagebreak

\section{Grundlagen}
	Dieser Teil der Arbeit gibt einen Überblick über sämtliche Bestandteile des Reinforcement Learnings. Dabei wird zunächst der wichtige mathematische Rahmen erläutert, der als Markov-Entscheidungsprozess verstanden wird. Aus diesem Rahmen lässt sich ein generisches Agent-Umwelt-Interface ableiten, auf welches eingegangen wird, um fundamentale Bestandteile, wie Belohnungen, Episoden, Gewinne und Nutzenfunktionen zu erläutern. 
\par
Abgerundet wird der Grundlagenteil mit der Auseinandersetzung von Strategien und dem Streben nach Optimalität, sowie der Darstellung des sog. \textit{Exploration-Exploitation-Dilemmas}.
	\subsection{Markow Entscheidungsprozess}
	\par 
Die Umwelt wird in den allermeisten Fällen als Markow-Entscheidungsprozess (\textit{Markov Decision Process, MDP}) definiert. //TODO Als \textit{MDP} versteht sich die Formalisierung von sequentiellen Entscheidungsproblemen, bei denen eine Entscheidung nicht nur die sofortige Belohnung beeinflusst, sondern auch alle Folgezustände und somit auch alle zukünftigen Belohnungen (S. 47). Zudem bieten sie den mathematischen Rahmen für das \textit{Reinforcement Learning} Problem, um z.B. Beweise über das Konvergenzverhalten eines Algorithmus hin zu einer optimalen Strategie führen oder andere theoretische Aussagen treffen zu können. Außerdem müssen Probleme die als \textit{MDP} definiert werden zugleich die Markow-Eigenschaft erfüllen, die von essentieller Bedeutung ist und in Kapitel X näher erläutert wird.
\par 

\begin{figure}[H]
    \centering
    \includegraphics[height=150px]{images/agentUmweltInterface.png}
    \caption{Agent-Umwelt Interface}
\end{figure}


Der Agent interagiert mit dem \textit{MDP} jeweils zu diskreten Zeitpunkten $t = 0, 1, 2, 3, \dots$. \\
Zu jedem Zeitpunkt $t$ beobachtet der Agent den Zustand seiner Umgebung $S_t \in \mathcal{S}$ und wählt aufgrund dessen eine Aktionen $A_t \in \mathcal{A}$. Als Konsequenz seiner Aktion erhält er einen Zeitpunkt später eine Belohnung $R_{t+1} \in \mathcal{R} \subset\mathbb{R} $ und stellt den Folgezustand $S_{t+1}$ fest. Das Zusammenspiel zwischen Agenten und MDP erzeugt also folgende Reihenfolge:
\[S_0, A_0, R_1, S_1, A_1, R_2, S_2, A_2, R_3, \dots\]

Wird einfach nur von \textit{MDPs} gesprochen, ist die endliche Variante (\textit{finite MDP}) gemeint, bei dem die Mengen der Zustände, Aktionen und Belohnungen ($\mathcal{S}, \mathcal{A}, \mathcal{R}$) eine endliche Anzahl an Elementen besitzen. In diesem Fall haben die zufälligen Variablen $R_t$ und $S_t$ wohl definierte diskrete Wahrscheinlichkeitsverteilungen, die nur von dem vorigen Zustand und vorigen Aktion abhängig sind (S.48). Die Wahrscheinlichkeit, dass die bestimmten Werte für diese Variablen $s' \in \mathcal{S}$ und $r \in \mathcal{R}$ eintreten, für einen bestimmten Zeitpunkt $t$ und dem vorigen Zustand $s$ und Aktion $a$, kann somit durch folgende Funktion beschrieben werden:

\[p(s',r \mid s,a) \doteq Pr\{S_t=s',R_t=r|S_{t-1}=s,A_{t_1}=a\},\]

für alle $s', s \in \mathcal{S}, r \in \mathcal{R}$ und $a \in \mathcal{A}(s)$. Diese Funktion p definiert die sog. Dynamiken (\textit{Dynamics}) eines \textit{MDP}. Sie ist eine gewöhnliche deterministische Funktion mit vier Parametern $p: \mathcal{S} \times \mathcal{R} \times \mathcal{A} \rightarrow [0,1]$. Das \glqq$\mid$\grqq{} Zeichen kommt ursprünglich aus der Notation für bedingte Wahrscheinlichkeiten, soll hier aber nur andeuten, dass es sich um eine Wahrscheinlichkeitsverteilung handelt für jeweils alle Kombinationen von $s$ und $a$:

\[ \sum_{s' \in \mathcal{S}} \sum_{r \in \mathcal{R}} p(s', r \mid s,a) = 1 \ \forall s \in \mathcal{S}, a \in \mathcal{A}(s)\]

Ist die Zustandsüberführungsfunktion nicht stochastisch, so ist $p$ immer nur für ein bestimmtes Triplet $(s,a,r)$ für jedes $s' \in \mathcal{S}$ gleich 1, für alle andere jeweils 0. Mit anderen Worten, wird im Zustand $s$ die Aktion $a$ gewählt, führt dies immer zu einem bestimmten Folgezustand $s’$. 
\par 

Das MDP Framework gilt als extrem flexibel und kann auf die unterschiedlichsten Probleme angewendet werden. Es bietet die nötige Abstraktion für Probleme, bei denen unter Vorgabe eines Ziels mittels Interaktionen gelernt wird. Dabei sind die Einzelheiten über eigentliche Ziel, die Zustände oder die Form des Agenten unerheblich, denn jedes zielgerichtete Lernen kann auf drei Signale reduziert werden, die zwischen dem Agenten und der Umwelt ausgetauscht werden. Ein Signal repräsentiert die Entscheidung, die der Agent getroffen hat (die Aktion), ein Signal repräsentiert die Basis, auf der er diese Entscheidung getroffen hat (der Zustand) und ein Signal definiert das zu erreichende Ziel (die Belohnung).



	\subsection{Markow-Eigenschaft und Zustandsmodellierung}
	Die Markov-Eigenschaft erhält ein eigenes Kapitel, da sie wichtig zum Verständnis dieser Arbeit ist und bei der Modellierung eines Reinforcement Learning Problems eine besondere Rolle spielt. Verbinden lässt sich dies sehr gut mit einem Einblick über die grundsätzliche Modellierung von Zuständen bei einem Reinforcement Learning Problem.

\begin{quote}
    The future is independent of the past given the present
  \end{quote}

Dieser Satz erscheint oft in der Literatur, wenn es um die Markov-Eigenschaft geht, so z.B. in den Arbeiten von \cite{Feldman2010}, \cite{kumar2014markov}, \cite{capela2019monogamy} und \cite{SaulMarkov}, oder auch in der Vorlesung der Stanford-Professorin Emma \cite{Brunskill}. Er fasst prägnant zusammen, was die Markov-Eigenschaft aussagt. Im Zusammenhang von MDPs lässt sich dieser Satz so übersetzen, dass ein Folgezustand nicht abhängig von Aktionen bzw. Zuständen in der Vergangenheit ist, sondern ausschließlich von dem aktuellen Zustand und der aktuell gewählten Aktion.
\par 
\cite{Sutton1998} sehen die Markov-Eigenschaft als Einschränkung für die Zustände und nicht für den Entscheidungsprozess als solches. 
Ausschlaggebend ist, dass der Zustand, auf dessen Basis der Agent seine Entscheidung trifft, alle notwendigen Informationen der Vergangenheit beinhaltet, die für die Zukunft relevant sind (S.49).
Die Umwelt ist somit nicht notwendigerweise gezwungen, Markov-konforme Zustände zu liefern. \cite{Brunskill} wählt aufgrund dessen die Bezeichnung \glqq Beobachtung\grqq{} (Observation $O_t$) als Feedback der Umwelt nach einer Aktion. Jene Beobachtungen können anschließend durch eine interne Repräsentation zu Markov-Zuständen verarbeitet werden, die dann dem Entscheidungsfinder zugrunde liegen.
\par
Folgendes Beispiel, basierend auf der Vorlesung von \cite{Brunskill}, liefert einen guten Einblick in die Zustandsmodellierung und der Problematik, die mit der Markov-Eigenschaft einhergeht.
\par 
\begin{figure}[H]
  \centering
  \includegraphics[height=200px]{images/2passagesDefault.png}
  \caption{ Zwei-Wege Beispiel zu der Markov-Eigenschaft}
  \label{fig:2-Wege-1}
\end{figure}

Gegeben ist ein beweglicher Roboter und eine Strecke mit zwei Korridoren. Der Roboter ist mit vier Sensoren ausgestattet, die jeweils eine Himmelsrichtung abdecken. Diese Sensoren sind in der Lage, angrenzende Wände zu erkennen und bilden den Zustand der Umwelt ab. Wahlweise ist der Zustand im Uhrzeiger definiert $\{N, O, S, W\}$, wobei 1 angibt, dass eine Wand erkannt wurde und 0, dass sich keine Wand in der unmittelbaren Nähe befindet. Es ergeben sich folglich 16 unterschiedliche Zustände, die der Agent unterscheiden und auf dessen Basis er Entscheidungen treffen kann (vier Aktionen: Fahrt in jeweils eine Richtung). Der Roboter soll sein Ziel erreichen, markiert mit einer Flagge, ohne dabei in eine der beiden Fallen zu navigieren.
\par
\begin{wrapfigure}{H}{0.5\textwidth}
  \begin{center}
  \includegraphics[height=200px]{images/2passagesStart.png}  \end{center}
  \caption{Zwei-Wege Beispiel Forts.}
  \label{fig:2-Wege-2}
\end{wrapfigure}

Eine potenzielle Startposition, wie in Abb. \ref{fig:2-Wege-1} dargestellt, liefert somit den Zustand $\{0, 1, 1, 1\}$. Angenommen der Agent hat gelernt in diesem Zustand Richtung Norden zu fahren, dann ist der Folgezustand ebenfalls $\{0, 1, 1, 1\}$. Schließlich erreicht er den ersten Korridor. Der westliche Sensor liefert folgerichtig 0 und der Zustand ist $\{0, 1, 0, 0\}$. Da der Agent nicht den ersten Korridor folgen darf, sondern dem zweiten, muss der Zustand $\{0, 1, 0, 0\}$ ebenfalls die Aktion \glqq nach Norden fahren\grqq{} auslösen. Das Besondere hier ist jedoch, dass der Zustand bei dem zweiten Korridor identisch mit dem Zustand bei dem ersten Korridor ist und der Agent somit keine Chance hat, zu unterscheiden, vor welchem er sich gerade befindet, siehe Abb. \ref{fig:2-Wege-2}. Er würde ebenfalls, wie schon bei dem ersten Korridor, weiter nach Norden und letztendlich in die Falle fahren.
\par 

Bezogen auf diesen Entscheidungsprozess ist die Modellierung der Zustände über den Sensorinput alleine nicht ausreichend, um die gestellte Aufgabe zu lösen. Die Kombination von Aufgabenstellung und dem Format der Zustände in dieser Form erfüllt insofern nicht die Markov-Eigenschaft, dass auf Basis der erkannten Zustände keine Möglichkeit besteht, die optimalen Entscheidungen zu treffen. 
\par

In der Theorie ist es jedoch möglich diesen Entscheidungsprozess als MDP umzumodellieren.
Dazu kann z.B. die gesamte Historie der Zustände und Aktionen gespeichert werden. Dadurch ist der Roboter in der Lage, zurückzuverfolgen, wo er sich zurzeit befindet. Obowhl dies rein theoretisch möglich ist, sollte diese Herangehensweise vermieden werden. Historien als Zustand für eine Entscheidung zu betrachten bedient zwar die Markov-Eigenschaft, ist allerdings in der Praxis nicht praktikabel, da der Zustandsraum auf diese Weise sehr schnell zu große Ausmaße annimmt.
\par 
Eine weitere Möglichkeit besteht darin, die Sensordaten als Beobachtungen der Umwelt zu betrachten und eine interne Repräsentation von Markov-Zuständen zu pflegen. Dabei bildet der Agent die Umwelt nach jeder erhaltenden Beobachtung suk­zes­si­ve nach, wodurch letztendlich ein Gesamtbild der Umgebung entsteht. Diese Form der Zustandsbildung bedarf jedoch zusätzlicher Algorithmen, die zwischen der Wahrnehmung des Agenten und dem jeweiligen RL-Algorithmus sitzen. 
\par 
Zudem ist das Lernen auf diese Weise auf eine bestimmte Umgebung festgelegt. Wird der Roboter mit einer neuer Welt konfrontiert, z.B. eine Welt mit drei Korridoren, so kann er Gelerntes nicht anwenden, weil seine interne Repräsentation invalide ist.
\par 
Letztendlich muss jedes Szenario zu Beginn genau untersucht werden, um zu beurteilen, ob Reinforcement Learning überhaupt auf dieses Problem anwendbar ist. Dabei spielt vor allem die Markov-Eigenschaft eine wichtige Rolle, die Grundvorraussetzung für alle RL-Algorithmen ist. Für eine Bewertung, ob optimales Verhalten durch gegebene Informationen erreicht werden kann,existieren jedoch keine festen Regeln oder eine Blaupause. Es muss auf Erfahrungswerte oder Untersuchungen zu dem Konvergenzverhalten und den Ergebnissen des gelernten Verhaltens zurückgegriffen werden, wie es auch in den Beispielen dieser Arbeit geschieht in dem Kapitel \ref{sec:praktischerTeil}.

	\pagebreak

	\subsection{Belohnungen und Zielstrebigkeit}
	Das Besondere an dem Reinforcement Learning ist das Belohnungssignal (\textit{Reward}), welches der Agent nach jeder Aktion erhält. Zu jedem diskreten Zeitpunkt wird dem Agenten eine Belohnung in Form einer einfachen Zahl $R_t \in \mathbb{R}$ zugestellt. Aufgabe eines jeden RL-Algorithmus ist es, die Summe aller gesammelten Belohnungen zu maximieren. Dabei ist entscheidend, dass der Fokus nicht ausschließlich auf die sofortigen Belohnungen gerichtet ist, sondern auf die erwartbare Summe aller Belohnungen über einen langen Zeitraum. Entscheidungen, die in der Gegenwart eine hohe sofortig Belohnung versprechen sind verführerisch, können sich aber in der Zukunft in Bezug auf den gesamten Prozess als suboptimal herausstellen. \cite[~S.53]{Sutton1998}
\par 
Eine Belohnungsfunktion wird in der Regel von einem Menschen definiert und hat den größten Einfluss darauf, wie der Agent sich verhalten soll. Die Festlegung von Belohnung bei bestimmten Events ist die einzige Möglichkeit, die der Agent hat, zu verstehen, welches Ziel er verfolgen soll. Somit ist die Modellierung der passenden Belohnungsfunktion zur korrekten Abbildung der eigentlichen Aufgabenstellung von gravierender Bedeutung.
\par 
Grundsätzlich gibt es zwei Ansätze, um eine Belohnungsfunktion zu formulieren. Verständlich werden diese durch ein Beispiel, bei dem ein Agent lernen soll, eine Partie Schach zu gewinnen. Die erste Möglichkeit besteht darin, dem Agenten ausschließlich eine Belohnung aufgrund des Spielausgangs zu geben. Er erhält +1, wenn er gewinnt, -1 bei einer Niederlage und 0 bei Unentschieden. Auf den ersten Blick erscheint dieser Ansatz trivial, ist aber die direkte Übersetzung des Ziels in eine Belohnungsfunktion. Die größte erwartbare Summe aller Belohnungen erzielt der Agent nur, wenn er lernt, das Spiel zu gewinnen. Größter Nachteil dieser Methode ist allerdings, dass der Agent keinerlei Hilfe oder Richtung bei dem Erkunden des Spiels erhält. Je größer der Zustands- und Aktionraum ist, desto länger braucht er um überhaupt einmal ein Spiel gewinnen zu können und zu lernen, welche Aktionen vorteilhaft sind und welche nicht.
\par 
Um dem entgegenzuwirken, werden dem Agenten bei der zweiten Möglichkeit feingranularere Belohnungen mitgeteilt, statt diese ausschließlich auf das Endresultat zu reduzieren. Bestimmte Belohnungen zeigen dann, ob der Agent seinem Ziel näher gekommen ist oder eine ungünstige Entscheidung getroffen hat. Zum Beispiel könnte dem Agenten eine hohe Belohnung von +10 gegeben werden, wenn er die gegnerische Dame aus dem Spiel nimmt. Es ist auch denkbar, dass jede Spielfeldkonstellation bewertet wird. Dieser Ansatz benötigt somit spezielles Vorwissen über das Problem und kann sich zugleich sehr negativ auf das Verfolgen des eigentlichen Ziels auswirken. Der Agent könnte zum Beispiel nur lernen in jedem Spiel die Dame des Gegners zu schlagen und dabei trotzdem immer die Partie zu verlieren.
\par 
Die korrekte Modellierung der Belohnungsfunktion hat somit eine besondere Bedeutung. \cite{Sutton1998} sind der Meinung, dass ein Schachagent nur angesichts des Spielaussgangs bewertet werden sollte und nicht aufgrund von Zwischenzielen wie z.B. dem Herausnehmen einer gegnerischen Spielfigur oder der Kontrolle über das Zentrum des Spielfelds (S.~53). 
\par 
Für eine korrekte Übersetzung der Aufgabenstellung zu einer geeigneten Belohnungsfunktion gibt es keine klaren, formalen Regeln. Ein*e Designer*in muss auf Erfahrungswerte und einen gewissen Grad an Kreativität zurückgreifen. Soll ein Agent z.B. ein Labyrinth durchlaufen und so schnell wie möglich hinausfinden, dann muss nach jeder Aktion eine negative Belohnung von -1 verteilt werden. Somit wird der Agent gezwungen, auf direktem Wege den Ausgang zu erreichen. Würde lediglich für das Erreichen des Ausgangs eine positive Belohnung vergeben werden, dann wäre die Summe aller Belohnungen für jede Abfolge von Aktionen gleich. Der Agent \glqq trödelt\grqq{}. Hat er durch Zufall aus dem Labyrinth gefunden, so könnte er bei weiteren Durchläufen keinen effektiveren Weg finden, denn für ihn haben alle Aktionsfolgen den gleichen Nutzen.
\par
Prinzipiell gilt, dass \glqq das Belohnungssignal dazu dient, dem Agenten mitzuteilen \textit{was} er erreichen soll, nicht \textit{wie} er es erreichen soll\grqq{} \cite[S.~54]{Sutton1998}.

	\subsection{Gewinn und Episoden}
	In Kapitel 2.1 wurde gezeigt, dass die Interaktion eines Agenten mit seiner Umwelt als bestimmte Abfolge beschrieben werden kann \eqref{eq:episode}. In ihr werden letztendlich alle Triple von Zustand, ausgeführter Aktion aufgrund dieses Zustands und anschließende Belohnung chronologisch aufgezeichnet. Ist diese Reihenfolge endlich, so wird sie auch als Episode (\textit{Episode}) bezeichnet. Eine Episode fasst somit alle Informationen zusammen, die ein Agent erlebt, während er von einem beliebigen Startzustand aus anfängt die Umwelt zu erkunden. Das Ende einer Episode wird durch das Erreichen eines beliebigen Zielzustands erreicht. Ist eine Episode zu Ende, dann wird das Szenario zurückgesetzt und der Agent startet erneut im Startzustand. Episoden sind komplett unabhängig voneinander und erzeugen Abfolgen, die nicht durch vorrige Episoden beeinflusst sind.
\par
Bisher wurde erwähnt, dass das Ziel eines Agenten sei, die Summe der zu erwartenden Belohnungen zu maximieren. Formal betrachtet, versucht er somit die Sequenz der Belohnungen, die er nach dem Zeitpunkt $t$ erhält, den sog. erwarteten Gewinn (\textit{Return}), zu maximieren. Im einfachsten Fall sieht $G_t$ wie folgt aus, wobei $T$ der finale Zeitstempel ist \cite[S.55]{Sutton1998}:

\begin{equation}\label{eq:simpleReturn}
    G_t = R_{t+1} + R_{t+2} + R_{t+3} + \dots + R_{T}
\end{equation}

Bei episodialen Problemen lässt sich der Gewinn durch diese Addition von nachfolgenden Belohnungen für jeden Zeitpunkt $t$ ermitteln. Grund hierfür ist, dass wärend der Berechnung, nach Abschluss der Episode, alle Belohnungen bekannt sind. 
\par 
Jedoch existieren auch Probleme, die keine Endzustände definiert haben und daher einen sog. unendlichen Zeithorizont (\textit{infinite horizon}) besitzen. Sie lassen sich nicht in natürliche Sequenzen unterteilen und werden auch mit \glqq kontinuierlich\grqq{} betitelt, wobei dadurch ausschließlich beschrieben wird, dass die Interaktion zwischen Agenten und Umwelt kein definiertes Ende besitzt, die Zeitstempel sind weiterhin diskret. Folglich ist $T=\infty$, was wiederum bedeutet, dass der Gewinn unendlich ist.
\par 
Um diese kontinuierlichen und die zuvor beschriebenen episodialen Aufgaben im Bezug auf den Gewinn zu vereinheitlichen, wird das Konzept der Diskontierung (\textit{discounting}) verwendet. Dabei gibt der Parameter $\gamma$, $0\leq \gamma \leq 1$, Auskunft darüber, wie die Gewichtung zwischen sofortigen und zukünftigen Belohnungen verteilt ist. Der zukünftige diskontierte Gewinn, der durch die Aktion $A_t$ maximiert werden soll, berechnet sich somit wie folgt \cite[S.55]{Sutton1998}:

\begin{equation}\label{eq:discountedReturn}
    G_t = R_{t+1} + \gamma R_{t+2} + \gamma^2 R_{t+3} + \dots  = \sum_{k=0}^\infty{\gamma^k R_{t+k+1}}
\end{equation}
Eine wichtige Erkenntnis ist, dass Gewinne aufeinanderfolgender Zeitpunkte in Verbindung stehen. Vor allem Algorithmen, die nach jedem Zeitstempel updaten, profitieren von dieser Eigenschaft. Sie verwenden den geschätzten Gewinn des Folgezustands, also $G_{t+1}$, zur Berechnung von $G_t$, dem geschätzten Gewinn des aktuellen Zustands. Dieses Verfahren, bei dem ein Schätzwert aufgrund eines anderen Schätzwertes aktualisiert wird, wird auch als \textit{bootstrapping} bezeichnet.  
\par 
Durch simple Umformung wird der Zusammenhang von Gewinnen deutlich \cite[S.55]{Sutton1998}:

\begin{equation}\label{eq:successiveReturn}
    \begin{aligned}
    G_t &= R_{t+1} + \gamma R_{t+2} + \gamma^2 R_{t+3} + \gamma^3 R_{t+4} + \dots \\
    &= R_{t+1} + \gamma (R_{t+2} + \gamma R_{t+3} + \gamma^2 R_{t+4} + \dots)  \\
   & = R_{t+1} + \gamma G_{t+1}
    \end{aligned}
\end{equation}
Ist $\gamma = 0$, dann wählt der Agent seine Aktionen ausschließlich aufgrund der sofortigen Belohnung $R_{t+1}$. Je näher $\gamma$ an 1 ist, desto \glqq weitsichtiger\grqq{} wird der Agent, da der Gewinn für den Zeitpunkt $t$ sich zusätzlich aus zukünftigen Belohnungen zusammensetzt. $\gamma = 1$ führt zu der gleichen Summe wie \eqref{eq:simpleReturn} und wird bei Problemen bestimmt, die Episoden erzeugen. Dadurch trifft der Agent seine Entscheidungen immer aufgrund jeglicher Konsequenzen in der Zukunft bzw. bis zum Ende der jeweiligen Episode. Um zu erreichen, dass die unendliche Summe in \eqref{eq:discountedReturn} bei kontinuierlichen Aufgaben einen endlichen Wert annimmt, muss $\gamma < 1$ gegeben sein.
\par 
Probleme mit unendlichem Zeithorizont können durch die Vergabe einer künstlichen Schranke zu einer episodialen Aufgabe umformuliert werden. Denkbar z.B. durch die Festlegung der maximalen Anzahl an Aktionen oder besuchten Zustände. 
\par 
//TODO TD-Episodic tasks?! Weglassen?
\par
Die Algorithmen der Monte-Carlo-Methoden, die in Kapitel \ref{sec:MC} vorgestellt werden, können ausschließlich auf Basis von Episoden lernen. Jedoch existieren auch Methoden, wie das Temporal-Difference-Learning, siehe Kapitel \ref{sec:TD}, die neben dem episodialen Lernen, zusätzlich in der Lage sind, mit kontinuierlichen Aufgaben zurechtzukommen. 

\bibliographystyle{apacite}
\bibliography{quellen}
\pagebreak

\end{document}
