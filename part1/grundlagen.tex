Dieser Teil der Arbeit gibt einen Überblick über sämtliche Bestandteile des Reinforcement Learnings. Dabei wird zunächst der wichtige mathematische Rahmen erläutert, der als Markov-Entscheidungsprozess verstanden wird. Aus diesem Rahmen lässt sich ein generisches Agent-Umwelt-Interface ableiten, auf welches eingegangen wird, um fundamentale Bestandteile, wie Belohnungen, Episoden, Gewinne und Nutzenfunktionen zu erläutern. 
\par
Abgerundet wird der Grundlagenteil mit der Auseinandersetzung von Strategien und dem Streben nach Optimalität, sowie der Darstellung des sog. \textit{Exploration-Exploitation-Dilemmas}.