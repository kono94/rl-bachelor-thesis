In dieser Arbeit wurden ausschließlich die tabularen Lernmethoden des Reinforcement Learnings betrachtet. Beispiele wie das Spielen von \textit{Atari}-Games mithilfe der gesammten Pixelkonfiguration \cite{dqn}, zeigen jedoch, dass das größte Potenzial in Methoden liegt, die mit Funktionsapproximation arbeiten. Vor allem der Bereich des \textit{Deep Reinforcement Learnings} ist dabei zu nennen, das Neuronale Netze verwendet, um die Nutzenfunktion zu approximieren.
\par
Die gesammelten Erkenntnisse über die Bestandteile und der Vorgehensweise bestimmter Algorithmen dient als exzellentes Fundament, um in einer kommenden (Master)-Arbeit auf komplexere Verfahren des Reinforcement Learnings einzugehen.